\chapter{Biblioteca Digital do Participa.br}

\section{Metadados de Participação Social}

\section{Versão de Testes da Biblioteca do Participa.br}

Dentre os ambientes de bibliotecas digitais testados, o Omeka foi escolhido como a infra-estrutura básica inicial para construção da biblioteca digital de participação social, isso se deve ao fato do Omeka ser bastante simples em questões de configuração e customização, pois utiliza tecnologias recentes como css3 e html5 em várias partes de seus temas. No entanto, outros ambientes podem ser testados e customizados dependendo das necessidades encontradas. Esse ambiente de testes se encontra no seguinte endereço: http://164.41.86.14:8080.
%
Foi definido um cabeçalho de acordo com o Modelo de Acessibilidade em Governo Eletrônico (eMAG), que é uma série de passos e recomendações para que os portais brasileiros sejam implementados de forma padronizada, de fácil implementação, coerente com as necessidades brasileiras e em conformidade com os padrões internacionais. A primeira versão do e-MAG foi disponibilizada para consulta pública em 18 de janeiro de 2005 e a versão 2.0 já com as alterações propostas, em 14 de dezembro do mesmo ano. Atualmente, o e-MAG encontra-se na versão 3.1 (última versão em dezembro de 2013). 
A partir disso, foi formulado um cabeçalho utilizando o formato e-MAG, conforme a imagem abaixo:

%Imagem Cabeçalho

Nesse menu, existem várias opções de acessibilidade, que ajudam os usuários portadores de necessidades acessar de forma mais facilitada a biblioteca de mecanismos formais de participação.

Para rodapé foi utilizado o padrão que há no portal Participa.br referências para o site oficial do Brasil, da secretaria geral da Presidência da República, além da referência ao software utilizado (Omeka) e a licença que está sendo utilizada.

%Imagem Rodapé

Foi pensado também um pequeno menu vertical, onde os usuários possam acessar as funções com mais rapidez. Esse menu pode ser visualizado na imagem abaixo:

%Imagem menu superior


Uma das vantagens do Omeka é a personalização desse menu. Através do painel de administração é possível adicionar e remover opções de maneira muito intuitiva, basta apenas que o link já esteja disponível (não aceita links quebrados). Nesse menu podemos apontar os links para visualizar todos os itens disponíveis na biblioteca (Ver itens), assim como especificar todos os itens de uma só coleção (Conselhos, Conferências e Ouvidorias). Uma página sobre, que vai conter todas as informações sobre a biblioteca digital e uma busca avançada, onde o usuário vai poder refinar suas buscas.

\subsection{Pagina Inicial}

A página inicial contém uma descrição sobre a biblioteca digital. Na figura a seguir é apresentado uma sugestão inicial de como funcionará a biblioteca de mecanismos formais de participação do Participa.br.

%Imagem tela principal
\begin{itemize}
\item \textbf{1 – Título da biblioteca} – Nessa parte é informado o título da biblioteca. Esse título é informado através do painel de administração da ferramenta.
\item \textbf{2 – Descrição da biblioteca} – Nessa parte é informada uma pequena descrição do objetivo daquela biblioteca para que o usuário possa se situar e verificar se aquele site atende as suas necessidades.
\item \textbf{3 – Menu Slider em jQuery} – Esse menu funciona como acesso rápido para que os usuários possam acessar os dados mais rapidamente, ele possui como funções básicas o acesso a Conferência, Conselhos e Ouvidorias, podendo ser alternado para os demais canais formais que possam existir.
\item \textbf{4 – Itens em Destaques} – Nessa local, podemos colocar aqueles canais que estão tendo reuniões naquele determinado dia/semana/mês. Essa parte pode ser configurada via painel de administração.
\item \textbf{5 – Exposição em Destaque} – Da mesma forma que a anterior, nesse local podemos colocar alguma exposição. Uma exposição é composta por itens que possuem o mesmo assunto, por exemplo, é possível juntar o Conselho Nacional de Saúde, a Conferência Nacional de Saúde e a Ouvidoria do Ministério da Saúde e criar uma exposição chamada Saúde e caso seja a semana nacional de Saúde, colocar aquela exposição em destaque.
\item \textbf{6 – Itens Adicionados Recentemente} – Nesse etapa, podemos visualizar os itens que acabaram de ser atualizados, isso também ajuda o usuário a perceber a movimentação do portal.
\item \textbf{7 – Listagem de um item na biblioteca} – Nessa listagem os usuários podem verificar o item brevemente. No entanto ao clicar no título é possível acessar o item e todas as suas informações.
\end{itemize}

\section{Integração de uma Biblioteca Digital no Participa.br}



