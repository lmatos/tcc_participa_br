\chapter{Sociedade e Participação Social}
\label{cap:sociedadepsocial}

Segundo BORDENAV (O QUE EH PARTICIPACAO) a participação popular em decisões governamentais é um dos principais mecanismos de consolidação da democracia de um país, como foi visto no ano de 1984 durante o movimento de Diretas Já \footnote{Movimento que buscava restabelecer as eleições diretas para presidente da República no Brasil.}. Outro fato histórico que teve grande participação social, foi a queda do muro de Berlin que aconteceu no ano de 1989, onde várias pessoas com intuito de passar o muro para ter uma vida melhor, derrubaram o mesmo. 

No ano de 2010 até os dias atuais ocorre no Oriente Médio a chamada Primavera Árabe, uma onda de manifestações e protestos da sociedade civil em diversos países daquele local. Esse evento é motivado pela grande corrupção, desemprego, autoritarismo por parte dos governos repressores desses países, desigualdade social, entre outros.  

As manifestação ocorridas no Brasil durante o período de junho do ano de 2013, possuiu os seus primeiros focos na cidade de São Paulo, onde os manifestantes protestaram contra o aumento da passagem de ônibus~\footnote{Maiores informações em: \url{http://g1.globo.com/sao-paulo/noticia/2013/06/entenda-os-protestos-em-sp-contra-aumento-das-tarifas-do-transporte.html}}. Essas manifestações tiveram uma grande repercussão social em diversos estados do Brasil, o qual motivou as pessoas desses estados a saírem para as ruas protestando contra a tarifa do transporte público de suas respectivas cidades. No entanto, a dimensão das manifestações e o número de pessoas na rua foi tão ampla, que outras pautas além da tarifa do transporte público foram colocadas em discussão, como a grande corrupção existente no Brasil, gastos abusivos e desnecessários durante a Copa do Mundo e Copa das Confederações, péssima qualidade dos serviços públicos como saúde, segurança pública, educação, transporte público, entre outros.

Desde o início da década de 1990 até os dias de hoje a participação popular em diversas frentes vem sendo amparada e institucionalizada em diversos locais do mundo dentro dos marcos das democracias representativas. (Políticas sociais locais e os desafios da participação citadina. Pedro Jacobi) No Brasil esse fato ocorreu ainda no ano de 1988, onde a Constituição de 88~\footnote{Disponível em: \url{http://www.planalto.gov.br/ccivil_03/constituicao/constituicao.htm}} define o Brasil como um Estado Democrático de Direito, atribuindo um novo modelo para gestão pública que trás a participação popular como o exercício pleno da cidadania, fazendo com que o cidadão tenha conscientização que ele faz parte da democracia, buscando melhorias do bem estar social para todos (Priscyla Mathias Scuassante). Segundo José Cláudio Rocha, a Constituição de de 1988 trouxe para o Brasil uma das legislações mais avançadas do mundo, no que diz respeito à proteção aos direitos humanos econômicos, sociais e culturais.

No entanto, apesar da Constituição de 1988 ter sido uma das mais avançadas, ainda fica claro que a população ainda desconhece os seus direitos, como por exemplo, poucos cidadões conhecem a existência desses mecanismos previstos na Constituição onde ela pode ``soltar a voz‘‘ para o governo através de críticas, sugestões, demandas, construção em políticas públicas, entre outros. 

Segundo (Pedro Bandeira Participação, Articulação de Atores Sociais e Desenvolvimento Regional) outro fator crucial para a participação atuante da sociedade civil na vida pública é o fato que ela corrobora para uma boa governança e para o desenvolvimento participativo. Além de permitir que a transparência das ações tomadas pelos governantes seja mantida e consequentemente permite o combate eficiente à corrupção no setor público.

A participação também é uma forma de ampliação da crítica da sociedade, visto que, quanto mais a população faz parte das ações tomadas pelo governo, proporcionalmente o seu senso de crítica sobre os acontecimentos se torna mais correto e justo, visto que a mesma conhece e contribui com as decisões que estão sendo tomadas pelos governantes. 

A participação popular se transforma no referencial de ampliação das possibilidades de acesso dos setores populares dentro de uma perspectiva de desenvolvimento da sociedade civil e de fortalecimento dos mecanismos democráticos, mas também de garantia da execução eficiente de programas de compensação social no contexto das políticas de ajuste estrutural e de liberalização da economia e de privatização do patrimônio do Estado.

Segundo (IPEA), atualmente a participação do cidadão na tomada de decisões políticas, na construção políticas públicas, demandando soluções para os problemas encontrados no cotidiano e na construção coletiva de soluções e inovações possa já é uma realidade. Hoje faz-se necessário adotar medidas para que todos meios de participação consolidados nos períodos atuais, seja compatíveis com a realidade encontrada no mundo tecnológico (redes sociais, participação social através do computador, conferências virtuais, entre outros).

\section{Mecanismos formais de participação}

Como visto na seção \ref{cap:sociedadepsocial}, a Constituição de 1988 garantiu o  princípio da participação social para todos os brasileiros, ou seja, a partir dessa data os brasileiros fazem parte na gestão, fiscalização e planejamento de políticas públicas (Trajano, 2011). Desde então houve uma crescimento de formas e instâncias de participação de todas as formas e tipos, presentes em todos os lugares do Brasil. Esses mecanismos ou canais, são lugares, onde o cidadão comum dialoga com membros da esfera do governo, tornando-os um lugar de encontro entre sociedade e estado. 

Esses mecanismos tem objetivo de permitir o acesso do cidadão ao processo decisório, esse mesmo processo que o mesmo não teve durante o período do regime militar. Nas seções abaixo contém uma pequena descrição de cada mecanismo formal de participação.

Conselho - As conferências nacionais é um dos principais canais de diálogo entre poder público e sociedade civil. Segundo a  últimos 10 anos, chegou a quase 90 em mais de 40 temas distintos, envolvendo níveis municipal, estadual e nacional, com grande capacidade de mobilização. 

- conferências

- ouvidorias

- mesas de dialogo

- comissão de politica

- fóruns 

- Ambiente virtual de participação

Existem outros mecanismos formais de participação, que apesar de não fazerem parte da constituição de 88, entende-se que também são locais em que há um diálogo entre a sociedade e membros da esfero do governo. É citado os dois principais mecanismos que fazem parte desse conceito:

- ONG

- Sindicatos

Existem vários outros mecanismos de participação não citados nesse trabalho, que apesar de não serem regidos em lei, que são utilizados como fonte de comunicação entre governo e sociedade.

\section{Participação Social Digital}

Tem-se tornado cada vez mais aceita, nos últimos anos, no Brasil, a ideia de que é necessário criar mecanismos que possibilitem participação mais direta da comunidade na formulação, no detalhamento e na implementação das políticas públicas.

Também são raras, no Brasil, organizações e instituições que congreguem e articulem diferentes segmentos da sociedade nessa escala territorial. 


\section{Exemplos de Participação Social Digital}

Como visto na seção anterior, a participação Social através de meios digitais surgiu recentemente com a propagação dos meios tecnológicos. Nessa seção será apresentado alguns exemplos de como funcionam as primeiras formas de participação social no Brasil e nos Estados Unidos, que influenciaram diretamente na arquitetura de funcionamento do portal de Participação Social Participa.br, discutido no capítulo \ref{cap:participabr} .

\subsection{A Primeira Experiência Brasileira: Gabinete Digital do RS}

Uma das primeiras experiências de participação social e democrática digital no Brasil foi através do Gabinete Digital do Rio Grande do Sul. O gabinete foi criado em 2011 e tem como seu principal objetivo “promover a cultura democrática e o fortalecimento da cidadania promovendo a eficiência e o controle social sobre o Estado, estruturando a relação do Governador com as diversas formas de escuta e participação através das redes digitais” (GABINETE DIGITAL, 2011).

Uma das funções do Gabinete Digital é criar canais de diálogo com a sociedade, através de ferramentas que são criadas como, por exemplo, monitoramento de obras, governador pergunta, governador responde, entre outras, o cidadão é estimulado a participar dando suas opiniões, debatendo ideias dadas por outros cidadãos, votando, acompanhando as ações que o governo toma, entre outras. Muitas dessas ferramentas que foram criadas ajudam ou já ajudaram o governo do Rio Grande do Sul na tomada de decisões.

O Gabinete Digital do Rio Grande do Sul criou algumas ferramentas, que são utilizadas pelo cidadão para participar do governo.

\subsubsection{Governador Pergunta}

Nessa ferramenta, o governador do estado elabora uma pergunta sobre uma determinada pauta que é definida a cada mês, por exemplo, a pauta do mês de dezembro é o transporte público, o governador solta uma pergunta “Como o transporte público pode ser melhorado?”. Durante duas semanas, as pessoas podem responder a pergunta dando sugestões, elogios, críticas, entre outros, em temas específicos sobre o assunto principal. As melhores propostas são organizadas pelos mediadores, para serem votadas. São dadas mais duas semanas para os participantes votarem nas melhores e mais importantes propostas. As cinquenta participações mais votadas são respondidas de volta pelo governador.

Os autores dessas cinquenta propostas são chamados pelo governo para debaterem suas ideias, além de ajudar a formular o documento que será utilizado para priorizar as ações tomadas pelas secretarias do governo do Rio Grande do Sul.
	Como forma de incentivar as pessoas participarem, o governo oferece prêmios diversos a pessoas que participam, como entrada em estádios de futebol. 

\subsubsection{Agenda Colaborativa}
	
Nessa ferramenta os participantes organizam a chegada do governador em seu município enviando demandas e necessidades específicas mais importantes. Após o envio dessas temáticas para a ferramenta, o governador transfere o seu ``local de trabalho‘‘  durante um dia para o município escolhido realizando um evento com a população. O governador coloca em pauta do evento, as temáticas com maiores relevâncias no evento.

\subsubsection{Governador Responde}

Essa ferramenta tem como objetivo criar um canal de comunicação entre os cidadãos e o governador do estado. Através dessa ferramenta, os participantes podem enviar perguntas ao governador sobre temas variados como saúde, transporte público, andamento de obras, entre outros. Dentro dessas perguntas existe um critério de votação, onde a pergunta mais votada no mês pelos outros participantes é escolhida para ser respondida pelo governador. O participante pode promover a sua pergunta através de outras redes sociais, para que outras pessoas possam visualizar e votar na pergunta, fazendo consequentemente que ela seja a mais votada.
	
\subsubsection{De olho nas obras}

Nessa ferramenta os cidadãos podem acompanhar os andamentos das obras de todo o governo do estado do Rio Grande do Sul. Nessa ferramenta, todas as obras são listadas, o usuário pode criar um cadastro para acompanhar uma obra em especifico. Quando o usuário deseja acompanhar o andamento de uma obra, ela apresenta todos os dados incluindo local da obra, seus respectivos custos, porcentagem de conclusão, uma possível data de término, secretária responsável e empresa contratada para realização.
	
Quando uma obra é selecionada é observada uma espécie de ``blog‘‘ , onde o governo coloca todas as etapas juntamente com fotos que detalham a situação do andamento da obra. A cada uma dessas etapas o cidadão que está acompanhando o andamento pode dar alguma sugestão, elogio ou crítica sobre o andamento da mesma, além de marcar a situação da obra como importante (como o ``curtir‘‘  do \textit{Facebook}). 

\subsection{\textit{Stabilize the Debt}: Controle o Gasto do Governo Americano}

O \textit{Stabilize the debt} é um simulador criado pelo \textit{The Committe for a Responsible  Federal Budget  (CRFB)}  em  maio  de  2010.  Nessa  ferramenta  os  participantes  são  incentivados  a controlarem  o  débito  fiscal  dos  Estados  Unidos,  controlando  os  gastos  relativos  as  políticas públicas, gastos com manutenção de tropas no Afeganistão e Iraque, continuação do financiando do programa \textit{Joint Strike Fighter (JSF F35)}, entre outros. O participante tem que manter o débitodos Estados Unidos em 60% do PIB até o ano de 2021. 

O simulador utiliza caixas de seleção em que cada tópico, onde o usuário decide se diminui os gastos  de  um  determinado  tópico.  Através  de  uma  biblioteca  \textit{javascript  budget\_calculator.js},  as seleções do usuário são automaticamente calculadas no momento da escolha e o resultado e gráfico são atualizados.


