\chapter{Sociedade e Participação Social}
\label{cap:sociedadepsocial}

Bordenave (\citeyear{bordenave1983participacao}) diz que a sociedade civil\footnote{Sociedade civil é composta por cidadãos, os coletivos, os movimentos sociais institucionalizados ou não institucionalizados, suas redes e suas organizações} em decisões governamentais é um dos principais mecanismos de consolidação da democracia de um país. Isso foi visto no ano de 1984 durante o movimento de Diretas Já \footnote{Movimento que buscava restabelecer as eleições diretas para presidente da República no Brasil.}. Outro fato histórico que teve grande participação social, foi a queda do muro de Berlim que aconteceu no ano de 1989, onde várias pessoas com intuito de passar o muro para ter uma vida melhor, derrubaram o mesmo. 

A participação também é uma forma de ampliação da crítica social da sociedade. Quanto mais a população faz parte das ações tomadas pelo governo, proporcionalmente o seu senso de crítica sobre os acontecimentos se torna mais maduro, já que a mesma conhece e contribui com as decisões que estão sendo tomadas pelos governantes. Esse aumento no senso crítico da população foi um dos principais fatores do acontecimento ocorrido no ano de 2010, que se estende até os dias atuais, chamado de Primavera Árabe. Esse evento ocorre no Oriente Médio, onde há uma série de manifestações e protestos da sociedade civil em diversos países daquela região. Os manifestantes motivados pela grande corrupção, desemprego, autoritarismo por parte dos governos repressores desses países, desigualdade social, entre outros, se uniram e manifestaram contra as ações o governo. Em alguns países como o Egito os governantes foram depostos, já em outros, como Líbia e Síria os manifestantes e governantes entraram em uma guerra civil.  

Desde o início da década de 1990 até os dias de hoje a participação popular em diversas frentes vem sendo amparada e institucionalizada em diversos países pelo mundo, tornando-os verdadeiras democracias representativas. (Políticas sociais locais e os desafios da participação citadina. Pedro Jacobi) No Brasil esse fato ocorreu ainda no ano de 1988, onde a Constituição \citeyear{cf}~\footnote{Disponível em: \url{http://www.planalto.gov.br/ccivil_03/constituicao/constituicao.htm}} define o Brasil como um Estado Democrático de Direito, atribuindo um novo modelo para gestão pública que trouxe a participação popular como o exercício pleno da cidadania, fazendo com que o cidadão tenha conscientização que ele faz parte da democracia, buscando melhorias do bem estar social para todos \cite{scuassante2009constituicao}. 

A Constituição de (\citeyear{cf}) prevê três formas de participação social. Essas formas variam de acordo com cada poder da República \cite{mds2008participacao}, são eles:

\begin{itemize}
	\item \textbf{Poder Legislativo} - Os cidadãos participam através do voto. O voto tem como objetivo eleger pessoas que vão representar a sociedade na defesas de seus direitos. Como o voto é obrigatório, a participação através do Poder Legislativo acaba se tornando a principal forma de participação social;
	\item \textbf{Poder Judiciário} - A participação popular ocorre quando pessoas da sociedade civil são convocadas para participar de um júri, geralmente julgando crimes contra  que são dolosos contra a vida (Juri Popular); e
	\item \textbf{Poder Executivo} - A participação popular acontece por meio dos mecanismos formais que são regidos em lei (conselhos, conferências, mesas de diálogo, fóruns, etc.). Esses mecanismos serão discutidos mais adiante na seção \ref{sec:mecformalpart}. A participação através desse poder acontece de forma mais direta com os gestores de políticas públicas.
\end{itemize}

Segundo Rocha (\citeyear{rocha2008popular}), a Constituição de de 1988 trouxe para o Brasil uma das legislações mais avançadas do mundo, no que diz respeito à proteção aos direitos humanos econômicos, sociais e culturais. No entanto, apesar da Constituição de 1988 ter sido uma das mais avançadas, ainda fica claro que a população ainda desconhece o poder de seus direitos, como por exemplo, muitos cidadãos ainda desconhecem a existência desses mecanismos previstos na Constituição onde ela pode mostrar sua opinião para o governo federal.

Bandeira (\citeyear{bandeira1999articulacao}) mostrou outro fator crucial sobre a participação atuante da sociedade civil na vida pública. Ela ajuda para uma boa governança e para o desenvolvimento participativo, consequentemente isso permite a garantia da transparência sobre ações tomadas pelos governantes, juntamente com o combate à corrupção no setor público.

A partir do primeiro mandato do presidente Luiz Inácio Lula da Silva em 2003, o governo passou a dar mais importância para a participação social, que se tornou uma principais formas de governar. Uma prova disso é que no mesmo ano de seu primeiro mandato foram criados vários novos conselhos, e os que existiam foram reformulados para permitir que a sociedade civil estivesse mais presente nos mesmos. O governo também percebeu que a sociedade também é essencial na concepção, execução e manutenção de novas políticas públicas, logo foram criados novos ambientes de participação onde permite que a participação da sociedade civil.

Atualmente a participação da sociedade civil na tomada de decisões políticas, construção políticas públicas, demandando soluções para os problemas encontrados no cotidiano e na construção coletiva de soluções possa já é uma realidade. Hoje faz-se necessário adotar medidas para que todos meios de participação consolidados nos períodos atuais, seja compatíveis com a realidade encontrada no mundo tecnológico (redes sociais, participação social através do computador, conferências virtuais, entre outros).

Nesse trabalho, destacaremos apenas a participação da sociedade no Poder Executivo, já que o foco desse trabalho é trazer as informações presentes nos mecanismos formais de participação já existentes ou não, para uma plataforma social digital para consulta de todos os mecanismos.

\section{Mecanismos Formais de Participação Regidos em Lei}
\label{sec:mecformalpart}

Como visto na seção \ref{cap:sociedadepsocial}, a Constituição de 1988 garantiu o  princípio da participação social para todos os brasileiros, ou seja, a partir dessa data os brasileiros fazem parte na gestão, fiscalização e planejamento de políticas públicas \cite{trajano2011controle}. Desde então houve uma crescimento de formas e instâncias de participação de todas as formas e tipos, presentes em todos os lugares do Brasil. Esses formas de participação são lugares de debate onde cidadão comum dialoga com membros da esfera do governo, trazendo demandas, sugestões e críticas que agregam na formulação de novas políticas públicas. 

As manifestações ocorridas no Brasil durante o período de junho e julho no ano de 2013, tiveram seu início na cidade de São Paulo, onde os manifestantes protestaram contra o aumento da passagem de ônibus~\footnote{Maiores informações em: \url{http://g1.globo.com/sao-paulo/noticia/2013/06/entenda-os-protestos-em-sp-contra-aumento-das-tarifas-do-transporte.html}}. Essas manifestações tiveram uma grande repercussão social em diversos estados do Brasil, o qual motivou as pessoas desses estados a saírem para as ruas protestando contra a tarifa do transporte público de suas respectivas cidades. No entanto, a dimensão das manifestações e o número de pessoas na rua foi tão ampla, que outras pautas além da tarifa do transporte público foram colocadas em discussão, como a grande corrupção existente no Brasil, gastos abusivos e desnecessários durante a Copa do Mundo e Copa das Confederações, péssima qualidade dos serviços públicos como saúde, segurança pública, educação, transporte público, entre outros.

Essas manifestações trouxeram como reposta, a assinatura do Decreto 8243/\citeyear{decreto8243} pela presidente Dilma Rousseff. Esse decreto cria a Política Nacional de Participação Social, que estimula a participação da sociedade civil nas ações tomadas pelo governo.

Esse decreto promove a participação social na formulação, acompanhamento, monitoramento e avaliação das políticas públicas, além de descrever mecanismos de participação social nas três instâncias de governo: federal, estadual e municipal. Os mecanismos formais de participação tem como objetivo, permitir o acesso do cidadão ao processo decisório, esse mesmo processo que foi extinto durante o período do regime militar. Nas seções abaixo contém uma pequena descrição dos principais mecanismos formais de participação.

\subsection*{Conselhos}

Instância colegiada temática permanente, instituída por ato normativo, de diálogo entre a sociedade civil e o governo para promover a participação no processo decisório e na gestão de políticas públicas. Os Conselhos criam espaços próprios para discutição de pautas e interesses dos setores sociais, que buscam a melhoria da qualidade e a universalização da prestação de serviços.

Segundo a Secretaria Geral da Presidência da República (\cite{sgpr2010conselhos} existem no total trinta e cinco Conselhos, desses, dezenove Conselhos foram criados no ano 2003 a 2013 e outros dezesseis foram reformulados com o objetivo de ampliar a participação social.


\subsection*{Conferências}

A conferência tem como objetivo reunir membros do governo e sociedade civil para debater um tema em comum. Esse debate gera uma política pública que será o tema utilizados durante os próximos anos. As conferências funcionam como instâncias de construção de direitos ainda não reconhecidos pelo Estado, um exemplo disso é que o relatório da 8ª Conferência de Saúde serviu como insumo para a criação do SUS (Sistema Único de Saúde) na Constituição de 1988.

Essas conferências podem acontecer em diferentes etapas, sendo elas: regionais, municipais, distritais e estaduais, juntas elas ajudam a propor diretrizes sobre o tema da conferência e ter um direcionamento melhor para a etapa federal.

O IPEA \citeyear{avritzer2012conferencias} mostrou que as conferências nacionais fazem partes dos principais canais de diálogo entre poder público e sociedade civil e, nos últimos 10 anos, chegou a quase 90 em mais de 40 temas distintos, envolvendo níveis municipal, estadual e nacional, com grande capacidade de mobilização. A ampliação e o aumento no número de conselhos também são significativos na última década, configurando um mecanismo de participação referendado pela Constituição Federal, ainda que com diversos desafios de melhora de representatividade. \cite{solagna2014metodologias}

\subsection*{Ouvidorias}

A Ouvidoria é um canal de comunicação entre o cidadão e governo, com o objetivo de receber da sociedade civil sugestões, solicitações, reclamações, denúncias e elogios, a fim de melhorar a gestão pública de seu órgão.

\subsection*{Mesas de Dialogo}

É um mecanismo onde acontece debates e negociação entre participantes da sociedade civil e do governo diretamente envolvidos no intuito de prevenir, mediar e solucionar conflitos sociais;

\subsection*{Audiência Pública}

Mecanismo participativo de caráter presencial, consultivo, aberto a qualquer interessado, com a possibilidade de manifestação oral dos participantes, cujo objetivo é subsidiar decisões governamentais

\subsection*{Fóruns}

Mecanismo para o diálogo entre representantes dos conselhos e comissões de políticas públicas, no intuito de acompanhar as políticas públicas e os programas governamentais, formulando recomendações para aprimorar sua intersetorialidade e transversalidade; mecanismo para o diálogo entre representantes dos conselhos e comissões de políticas públicas, no intuito de acompanhar as políticas públicas e os programas governamentais, formulando recomendações para aprimorar sua intersetorialidade e transversalidade;

\subsection*{Consulta Pública}

A consulta pública é um mecanismo caráter consultivo, aberto a para a participação do público em geral. Ele tem como objetivo principal, receber contribuições sobre determinado assunto, sendo elas físicas (documento escrito) ou digital (ambiente de participação). Essas consultas públicas ajudam na construção de novas políticas públicas.

\section{Outros Mecanismos Formais de Participação}

Existem vários outros mecanismos de participação não citados na seção acima, que são utilizados como fonte de comunicação entre governo e sociedade, porém não são especificados formalmente no DECRETO. Apesar de não fazerem parte do decreto, entende-se que também são importantes serem citados, já que a Constituição Federal de 1988 promoveu a organização e participação social como um direito. Esses mecanismos criam locais em que o direito de participação é garantido através do diálogo entre a membros da sociedade civil e membros da esfera do governo federal. Abaixo citamos os dois principais mecanismos que fazem parte desse conceito:

\subsection*{ONG}

As ONG (\textbf{O}rganizações \textbf{N}ão \textbf{G}overnamentais), também chamadas de Organizações da Sociedade Civil são associações privadas sem fins lucrativos, criadas a partir de um interesse em comum. Essas associações, apesar de não lucrarem com suas ações, desenvolvem atividades que complementam as ações do Estado.

Atualmente segundo estudo FASFIL \cite{ibge2012ongs}, mostram que existem mais de 290 mil Organizações da Sociedade Civil em funcionamento no Brasil no ano de 2012, muitas delas surgidas após a constituição de 1988.

\subsection*{Sindicatos}

Os sindicatos são associações de defesa sobre assuntos em comum de determinado grupo social. Os principais sindicatos reúnem pessoas para tratar sobre assuntos de cunho econômico (conhecidos como sindicatos patronais) ou trabalhista (trabalhadores, transportes, metalúrgicos, policiais, professores, médicos, etc).

Os sindicatos têm como objetivo principal a defesa dos interesses econômicos, profissionais, sociais e políticos dos seus associados. 

\section{Ambiente Digital de Participação Social}

Tem-se tornado cada vez mais aceita, nos últimos anos, no Brasil, a ideia de que é necessário criar mecanismos que possibilitem ampliar a participação social, além de torna-lá mais próxima comunidade, seja na formulação, no detalhamento e na implementação das políticas públicas.

Segundo o \cite{decreto8243} o ambiente digital para participação social é caracterizado por um tipo mecanismo de interação social que utiliza tecnologias de informação e de comunicação, em especial a internet, para promover o diálogo entre administração pública federal e sociedade civil. 

\section{Exemplos de Participação Social Digital}

Como visto na seção anterior, a participação Social através de meios digitais surgiu recentemente com a propagação dos meios tecnológicos. Nessa seção será apresentado alguns exemplos de como funcionam as primeiras formas de participação social no Brasil e nos Estados Unidos, que influenciaram diretamente na arquitetura de funcionamento do portal de Participação Social Participa.br, discutido no capítulo \ref{cap:participabr}.

\subsection{A Primeira Experiência Brasileira: Gabinete Digital do RS}

Uma das primeiras experiências de participação social e democrática digital no Brasil foi através do Gabinete Digital do Rio Grande do Sul. O gabinete foi criado em 2011 e tem como seu principal objetivo ``promover a cultura democrática e o fortalecimento da cidadania promovendo a eficiência e o controle social sobre o Estado, estruturando a relação do Governador com as diversas formas de escuta e participação através das redes digitais.‘‘ \cite{gabinete2012participacao}.

Uma das funções do Gabinete Digital é criar vários canais de diálogo com a sociedade. Esses canais foram criados através de ferramentas participativas, por exemplo, monitoramento de obras, governador pergunta, governador responde, entre outras, o cidadão é estimulado a participar dando suas opiniões, debatendo ideias dadas por outros cidadãos, votando, acompanhando as ações que tomadas pelo governo, entre outras. Muitas dessas ferramentas que foram criadas ajudam ou já ajudaram o governo do Rio Grande do Sul na gestão de políticas públicas.

O Gabinete Digital do Rio Grande do Sul criou algumas ferramentas, que são utilizadas pelo cidadão para participar do governo.

\subsubsection*{Governador Pergunta}

Nessa ferramenta, o governador do estado elabora uma pergunta sobre uma determinada pauta que é definida a cada mês, por exemplo, a pauta do mês de dezembro é o transporte público, o governador solta uma pergunta ``Como o transporte público pode ser melhorado?‘‘. Durante duas semanas, as pessoas podem responder a pergunta dando sugestões, elogios, críticas, entre outros, em temas específicos sobre o assunto principal. As melhores propostas são organizadas pelos mediadores, para serem votadas. São dadas mais duas semanas para os participantes votarem nas melhores e mais importantes propostas. As cinquenta participações mais votadas são respondidas de volta pelo governador.

Os autores dessas cinquenta propostas são chamados pelo governo para debaterem suas ideias, além de ajudar a formular o documento que será utilizado para priorizar as ações tomadas pelas secretarias do governo do Rio Grande do Sul. Como forma de incentivar as pessoas participarem, o governo oferece prêmios diversos a pessoas que participam, como entrada em estádios de futebol. 

\subsubsection*{Agenda Colaborativa}
	
Nessa ferramenta os participantes organizam a chegada do governador em seu município enviando demandas e necessidades específicas mais importantes. Após o envio dessas temáticas para a ferramenta, o governador transfere o seu ``local de trabalho‘‘  durante um dia para o município escolhido realizando um evento com a população. O governador coloca em pauta do evento, as temáticas com maiores relevâncias no evento.

\subsubsection*{Governador Responde}

Essa ferramenta tem como objetivo criar um canal de comunicação entre os cidadãos e o governador do estado. Os participantes enviam perguntas ao governador sobre temas variados como saúde, transporte público, andamento de obras, entre outros. Dentro dessas perguntas existe um critério de votação, onde a pergunta mais votada no mês pelos outros participantes é escolhida para ser respondida efetivamente pelo governador. Caso seja de interesse, o participante pode promover a sua pergunta através de outras redes sociais.
	
\subsubsection*{De olho nas obras}

Nessa ferramenta os cidadãos acompanham o andamento das obras de todo o governo do estado do Rio Grande do Sul. Quando a mesma é acessada, ela apresenta em lista todas as obras do estado. O usuário pode criar um perfil no sistema caso deseje acompanhar uma obra em especifico, apresentando todos os dados incluindo local da obra, seus respectivos custos, porcentagem de conclusão, uma possível data de término, secretária responsável e empresa contratada para realização.
	
Quando uma obra é selecionada é observada uma espécie de blog, onde o governo mostra as etapas dessa obra, juntamente com fotos que detalham a situação encontrada no momento. A cada uma dessas etapas o cidadão que está acompanhando o andamento pode dar alguma sugestão, elogio ou crítica sobre o andamento da mesma, além de marcar a situação da obra como importante (como o ``curtir‘‘  do \textit{Facebook}). 

\subsection{\textit{Stabilize the Debt}: Controle o Gasto do Governo Americano}

O \textit{Stabilize the debt} é um simulador criado pelo \textit{The Committe for a Responsible  Federal Budget  (CRFB)}  em  maio  de  2010.  Nessa  ferramenta  os  participantes  são  incentivados  a controlarem  o  débito  fiscal  dos  Estados  Unidos,  controlando  os  gastos  relativos  as  políticas públicas, gastos com manutenção de tropas no Afeganistão e Iraque, continuação do financiando do programa \textit{Joint Strike Fighter (JSF F35)}, entre outros. O participante tem que manter o débitodos Estados Unidos em 60% do PIB até o ano de 2021. 

O simulador utiliza caixas de seleção em que cada tópico, onde o usuário decide se diminui os gastos  de  um  determinado  tópico.  Através  de  uma  biblioteca  \textit{javascript  budget\_calculator.js},  as seleções do usuário são automaticamente calculadas no momento da escolha e o resultado e gráfico são atualizados.


