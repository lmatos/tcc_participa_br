\chapter{Considerações Finais}

Nesse trabalho além das conferências, conselhos e ouvidorias, alguns outros canais de participação foram pesquisados para composição da versão inicial da biblioteca digital do Participa.br. Dentre essas alternativas estão as mesas de diálogo, as ONGs e os sindicatos. Porém, para essa primeira versão desta monografia só foram considerados conselhos, conferências e ouvidorias. Alguns pontos são esclarecidos.

As audiências públicas são importantes instrumentos formais de participação social no Brasil. Em estudo realizado pelo IPEA (\citeyear{ipea2013audiencia}), consta um levantamento indicando que de 2004 a 2009 foram realizadas 203 processos de audiência pública. Dada a quantidade de itens, esse instrumento será considerado apenas para a segunda versão deste trabalho.

Sobre mesas de diálogo, não foram encontrados bons exemplos, nem muita legislação a esse respeito, mas apenas pequenos textos soltos e poucas notícias de ocorrência de mesas de diálogos são encontrados na Internet. Também não há uma frequência dessas mesas durante um período e a maioria delas ocorre durante uma manifestação, acontecimento (desastres, alguns impasses de terras, entre outros), o que tornou mais difícil a pesquisa.

Em relação às ONGs, considera-se que a catalogação de todas as ONG’s é desnecessária nesse primeiro momento, pois a quantidade é muito grande  \cite{ibge2012ongs} \footnote{Segundo a pesquisa FASFIL em 2010, realizada pelo IBGE, no ano de 2010 já haviam cerca 290,7 mil Fundações Privadas e Associações sem Fins Lucrativo o que representa já 5,2\% de todas as associações com lucro ou sem lucro. E a cada 5 anos o crescimento em número de ONG’s é praticamente acrescido em 50\%.}. Da mesma forma, o número de sindicatos é muito elevado (cerca de mais de 15000 no ano de 2011) e a estratégia adotada foi adiar a documentação desses mecanismos após a biblioteca digital estar em pleno funcionamento.

Já nas conferências, não há a utilização de ferramentas digitais para ajudar ou amplificar a sua atuação. Uma das grandes dificuldades é a gestão da informação e do processo conferencial. Como são inúmeras conferências municipais e estaduais, o processo de sistematização e organização das informações sobre participação social se faz necessário. 

Em princípio, a biblioteca digital proposta pode ser naturalmente referenciada dentro do Portal do Participa, sem que o usuário perceba que se trata de um elemento com arquitetura diferente do Noosfero, principalmente se a implantação final vier a ser feita com uso do Greenstone ou com o Omeka. Por enquanto o protótipo gerado está em uma fase de testes \footnote{Disponível no endereço: \url{http://164.41.86.14:8080}}.

No que se refere à montagem de uma biblioteca digital para distribuição, parece que o Greenstone \footnote{Disponível no endereço: \url{http://http://www.greenstone.org/}} ou o DSpace \footnote{Disponível no endereço: \url{http://http://www.dspace.org/}} são as soluções de software mais adequadas, já que permitem customização do ambiente para distribuição nos moldes do que foi planejado no contexto deste projeto.


\section{Cronograma}

Existem atividades planejadas no escopo deste trabalho em curto e médio prazo. Essas atividades incluem o desenvolvimento da biblioteca do Participa.br, assim como o desenvolvimento do plugin para consumo dos conteúdos presentes nos mecanismos formais de participação. Um dos pontos a ser considerado para o TCC2, foi a evolução do Noosfero para a versão 3 do arcabouço \textit{Ruby on Rails}, que vai demandar uma pequena aprendizagem para que seja possível criar o plugin do OAI-PMH. As atividades planejadas são:

\begin{enumerate}
\item Fechamento de um software para biblioteca digital;
\item Criação do primeiro protótipo funcional da biblioteca digital de participação;
\item Implementação de outros conteúdos (áudio, vídeo, imagens, etc) para a biblioteca digital;
\item Criação de um formulário para padronizar as informações dos mecanismos formais de participação;
\item Implementação do Plugin OAI-PMH para o Noosfero;
\item Adaptação da biblioteca para outros mecanismos formais;
\item Pesquisa e adaptação da biblioteca digital para uma biblioteca digital semântica (ontologias);
\item Escrita do TCC 2.
\end{enumerate}

 
\begin{table}[H]
\begin{center}
	\caption{Cronograma para atividades do TCC2}
    \begin{tabular}{ 'l | l | l | l | l | l | l | l' |}\thickhline
    \rowcolor[HTML]{BFBFBF}
    \multicolumn{1}{!{\vrule width 2pt}c!{\vrule width 1pt}}{\textbf{Atividade}} & 
\multicolumn{1}{!{\vrule width 0pt}c!{\vrule width 1pt}}{\textbf{Jul 2014}} & 
\multicolumn{1}{!{\vrule width 0pt}c!{\vrule width 1pt}}{\textbf{Ago 2014}} & 
\multicolumn{1}{!{\vrule width 0pt}c!{\vrule width 1pt}}{\textbf{Set 2014}} &
\multicolumn{1}{!{\vrule width 0pt}c!{\vrule width 1pt}}{\textbf{Out 2014}} & 
\multicolumn{1}{!{\vrule width 0pt}c!{\vrule width 1pt}}{\textbf{Nov 2014}} & 
\multicolumn{1}{!{\vrule width 0pt}c!{\vrule width 2pt}}{\textbf{Dez 2014}} \\ \noalign{\hrule height 3pt}
    1 & • & & & & & \\ \hline
    2 & • & & & & & \\ \hline
    3 & • & • & & & & \\ \hline
    4 & & • & • & & & \\ \hline
    5 & & • & • & • & & \\ \hline
    6 & & • & • & • & & \\ \hline
    7 & & & & • & • & • \\ \hline
    8 & • & • & • & • & • & • \\ \noalign{\hrule height 3pt}
    \end{tabular}
    \label{tab-cronograma}
\end{center}
\end{table}


Avaliamos que, durante o TCC 1 já foi feito o levantamento de parte das informações presentes nos principais mecanismos de formais de participação, além dos principais softwares de biblioteca digital de participação social, como foi visto no capítulo~\ref{cap:bibparticipabr}. No entanto, vale a pena resaltar que a Secretária Nacional de Juventude \footnote{Maiores informações em: \url{http://www.juventude.gov.br/}} está implementando uma biblioteca digital sobre dados da juventude utilizando o software DSpace, além de estarem migrando todo o seu conteúdo para o Noosfero. Esse fato pode ajudar na decisão final sobre a utilização do DSpace como software para biblioteca digital de participação social, já que assim garante feito o desenvolvimento colaborativo entre as duas bibliotecas.
 

