\chapter{Bibliotecas Digitais}

Segundo uma biblioteca digital não é simplesmente uma coleção digital com ferramentas para manutenção das informações. Uma biblioteca digital é composta por um ambiente que possui coleções, serviços e pessoas apoiando um ciclo de vida de disseminação, uso e preservação do dado, informação e conhecimento.


\section{Metadados para catalogação de items}
%REFERENCIAR
Dentro da biblioteconomia, um metadado é um dado estruturado que compartilha diversas características similares para a catalogação. Os metadados descrevem as características de um determinado recurso informacional.

Metadados são utilizados para descrição de arquivos de documentos, imagens, vídeos, páginas web, softwares e qualquer outro conteúdo que necessitam de uma busca e obtenção de forma mais refinada. %IAWWWW

%documents, pages, images, software, video and audio files, and other content objects for the purposes of improved navigation and retrieval.

%Metadados com XML
%Construção dos metadados
%Indexação (Section?)
	
\section{Padrões de Metadados}
%
Na Ciência da Informação, há diversos padrões especificados com esquemas de metadados. Cada padrão de metadado, deve possuir o nome do metadado (exemplo: dc.Creator), assim como seu significado. Existem vários formatos de metadados, no entanto dos principais metadados, podemos citar:

\begin{itemize}
\item \textbf{DC} (Dublin Core) é um padrão contendo 15 propriedades para descrição de itens. O nome Dublin se da por causa da conferência no ano de 1995 em Dublin, e o outro termo Core por descrever objetos com várias finalidades. \cite{dbcore2012set}. O DC é bastante flexível quanto ao seu uso e não é necessário especialização, podendo ser utilizado para vários contextos.

\item \textbf{AACR2x} (Anglo-American Cataloguing Rules) - 
\item \textbf{MARC} (Machine Readable Cataloging) - 
\end{itemize}

%Metadados de banda 1,2 e 3
%Figura exemplo de metadados

\section{Bibliotecas Digitais Semânticas}


