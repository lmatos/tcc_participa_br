\chapter{Bibliotecas Digitais}
%Introdução
Uma biblioteca digital é

\section{Metadados para catalogação de items}
%REFERENCIAR
Dentro da biblioteconomia, um metadado é um dado estruturado que compartilha diversas características similares para a catalogação. Os metadados descrevem as características de um determinado recurso informacional.
%Metadados com XML
%Construção dos metadados
%Indexação (Section?)
	
\section{Padrões de Metadados}
%
Na Ciência da Informação, há diversos padrões especificados com esquemas de metadados. Cada padrão de metadado, deve possuir o nome do metadado (exemplo: dc.Creator), assim como seu significado. Existem vários formatos de metadados, no entanto dos principais metadados, podemos citar:

\begin{itemize}
\item \textbf{DC} (Dublin Core) é um padrão contendo 15 propriedades para descrição de itens. O nome Dublin se da por causa da conferência no ano de 1995 em Dublin, e o outro termo Core por descrever vários objetos com várias finalidades \cite{dbcore2012set}.

\item \textbf{AACR2} (Anglo-American Cataloguing Rules) - 
\item \textbf{MARC} (Machine Readable Cataloging) - 
\end{itemize}

%Metadados de banda 1,2 e 3
%Figura exemplo de metadados

\section{Bibliotecas Digitais Semânticas}
%
Bibliotecas Digitais Semânticas utilizam...

\section{Levantamentos de Softwares para Bibliotecas Digitais}

Foi realizada uma primeira pesquisa a fim de verificar possíveis ferramentas que vão compor a bilioteca de participação social. Dentro dessas ferramentas, também foram investigadas ferramentas de bibliotecas digitais semânticas. Em função dessa investigação, foram identificados softwares de bibliotecas digitais que pudessem absorver a construção de metadados baseados em ontologias, além de ferramentas de apoio à apresentação de canais formais de participação social de forma didática. 
%
Para atender a essa necessidade, foi feito um levantamento a respeito e foram identificados os softwares que suportam bibliotecas digitais semânticas como  o JeromeDL\footnote{\url{http://www.jeromedl.org/}}, Simile\footnote{\url{http://simile.mit.edu/}} e DSpace\footnote{\url{http://www.dspace.org/}}. Também foram investigados os softwares GreenStone\footnote{\url{http://www.greenstone.org/}}, KEA\footnote{\url{http://www.nzdl.org/Kea}} e Omeka\footnote{\url{http://www.omeka.org/}} que apresentam uma forma didática para criação e apresentação dos dados sobre os mecanismos formais de participação. Todos os softwares listados acima, vão ser descritos a seguir.
%
É importante ressaltar que, no estágio atual dessa investigação, alguns desses softwares já estão instalados, mas ainda há a necessidade de realizar mais testes para identificar em que medida eles podem ser aproveitados na arquitetura de biblioteca digital semântica de participação social.Os softwares  de bibliotecas digitais GreenStone, KEA e Omeka foram instalados em ambientes de testes na infraestrutura do laboratório LAPPIS para a primeira fase da pesquisa. Os demais softwares DSpace, Simile e JeromeDL foram citados apenas como pesquisa para o prosseguimento do projeto.

%Ferramentas concorrentes de bibliotecas digitais
Para apresentação da biblioteca sobre os canais formais de participação, foram encontrados alguns softwares de bibliotecas digitais com propósitos similares, entre eles:

\begin{itemize}
%Simile
\item \textbf{Simile} (Semantic Interoperability of Metadata and Information in unLike Environments) é um projeto...
%JeromeDL
\item \textbf{JeromeDL} JeromeDL é uma Biblioteca Digital Semântica Social. Através da utilização de padrões de Web Semântica (como por exemplo: RDF e OWL), ele amplia as funcionalidades de busca e pesquisa dos itens contidos em uma biblioteca digital;
%Dspace
\item \textbf{DSpace} é uma ferramenta de bibliotecas digitais;
%KEA
\item \textbf{KEA} (Keyphrase Extraction Algorithm) fornece um conjulnto de ferramentas de análise e benchmarking de projetos de software livre.
%GreenStone
\item \textbf{GreenStone} é uma metodologia baseada em quatro passos: definição de referência utilizada, avaliação do software, qualificação de usuários específicos, e por último, seleção e comparação de software;
%Omeka
\item \textbf{Omeka}  é um projeto que fornece uma base de dados aberta com informações sobre licenças de software;
\end{itemize}

\section{Primeiros Testes com Ferramentas de Bibliotecas Digitais}

%O que é o Noosfero
\subsection{GreenStone}
\subsection{Omeka}

