\chapter{Noosfero}

%Contexto
 O Noosfero é um software livre para criação de redes sociais, que está disponível sob licença AGPL\footnote{Licença de software GNU Affero General Public License} V3, com o intuito de permitir que os usuários criem sua própria rede social livre e personalizada de acordo com suas necessidades.

A linguagem de programação Ruby e o arcabouço MVC Ruby on Rails foram utilizados para desenvolver o Noosfero. Essas tecnologias foram escolhidas pois a linguagem Ruby possui uma sintaxe simples, que facilita a manutenibilidade do sistema, característica importante em projetos de software livre que tendem a atrair colaboradores externos a equipe. Já o arcabouço Ruby on Rails influencia em maior produtividade graças a conceitos como \textit{convention over configuration} e DRY. Por esse motivo o Noosfero ``herda'' sua arquitetura, a qual é baseada no padrão arquitetural MVC, assim como os plugins que estendem suas funcionalidades.

%Arquitetura de plugins do Noosfero
A arquitetura do Noosfero permite a adição de novas funcionalidades através de plugins. Essa característica é interessante, pois colaboradores podem incorporar novas funcionalidades ao Noosfero, já que os plugins possuem o código isolado, mantendo o baixo acoplamento e alta coesão dos módulos do sistema.
%
Embora plugins sejam totalmente independentes do sistema alvo, no Noosfero os plugins são mantidos com o código principal para auxiliar no controle de qualidade do ambiente. Pensando nisso os plugins devem ter testes automatizados. Quando houver a necessidade de alterar o código do Noosfero, os testes dos plugins são executados para verificar se as mudanças não afetaram seu funcionamento~\footnote{\url{http://noosfero.org/Development/Plugins}}.
%
O funcionamento dos plugins é baseado no paradigma de orientação a eventos. O núcleo do Noosfero dispara um evento durante sua execução e os plugins interessados nesse evento saberão como tratar esse evento. Os eventos que são disparados pelo Noosfero são chamados de “hotspots”.