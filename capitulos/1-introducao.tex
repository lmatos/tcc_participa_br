\chapter{Introdução}

A participação popular pode ser compreendida como um processo no qual homens e mulheres se descobrem como sujeitos políticos, exercendo os direitos políticos \cite{almeida2011participacao}. Após 2003 o governo passou a enxergar essa iniciativa popular como um mecanismo de concepção, execução e manutenção de novas políticas públicas. Antigamente todas as medidas que era tomadas era definidas apenas por técnicos e ministros \cite{sgpr2010conselhos}.

Uma das forma de participação previstas em lei é através dos mecanismos formais de participação criados na Constituição de \citeyear{cf} e instituídos através de política pública no Decreto 8243/\citeyear{decreto8243}, que cria a Política Nacional de Participação Social.

Atualmente existem uma gama de mecanismos formais de participação, sejam eles conselhos, conferências, ouvidorias, mesa de diálogos, entre outros, todos eles possuem uma função semelhante: dar voz para o povo demandar, criticar, sugerir, elogiar, fazer políticas públicas, entre outros. 

Embora exista uma quantidade enorme de informações sobre participação social e mecanismos formais de participação, foi percebido que existem informações desatualizadas, em especial as que estão disponibilizadas em catálogos impressos e em sites. De fato, é evidente a dinâmica dos canais de participação e esse dinamismo provoca o surgimento constante de novidades, com novas definições de agendas de encontros, com mudanças de nomes de gestores participantes, de números de telefone e uma série de outras informações correlatas. Por isso, se faz necessário a criação de um canal centralizado, onde vão estar disponível todas as informações sobre os mecanismos formais de participação.

\section{Objetivos}

\subsection{Objetivos Gerais}

O objetivo desse trabalho de conclusão de curso é o desenvolvimento de um ambiente de consulta colaborativa, que funcione de maneira descentralizado onde cada instância é responsável pela manutenção de suas informações e que essas todas essas informações sejam centralizadas. 

\subsection{Questões de pesquisa}

As questões de pesquisa respondidas por este trabalho são:

\begin{itemize}
\item \textbf{QP1} - Como as bibliotecas digitais garantem a atualização das informações presentes nos mecanismos formais de participação social?
\item \textbf{QP2} - Bibliotecas digitais descentralizadas potencializam os mecanismos formais de participação social em rede?
\item \textbf{QP3} - Quais informações dos mecanismos formais de participação devem ser catalogadas?
\item \textbf{QP4} - Como o Noosfero pode ser integrado com uma biblioteca digital?
\item \textbf{QP5} - Como integrar as bibliotecas digitais com as ontologias de participação e metodologias propostas?
\end{itemize} 

Baseado nas perguntas acima, este trabalho busca solucionar os seguintes objetivos específicos: 

\subsection*{Objetivos Tecnológicos}

\begin{itemize}
\item \textbf{OT1} - Desenvolvimento de uma biblioteca digital para permitir a disponibilização das informações (textos, vídeos, imagens, áudio, mapas, entre outros) sobre os mecanismos formais de participação atualizadas e descentralizadas;
\item \textbf{OT2} - Adaptação do Noosfero\footnote{Mais informações em: \url{http://www.noosfero.org}}, uma ferramenta para criação de redes sociais para implementar o protocolo OAI-PMH. Esse protocolo vai garantir que as informações sobre os mecanismos formais de participação presentes nas bibliotecas digitais sejam visualizados na própria ferramenta Participa.br.
\item \textbf{OT3} - Ampliação da biblioteca digital para uma biblioteca digital semântica, utilizando como base ontologias e metodologias já conhecidas a fim de disponibilizar um ambiente mais completo para o usuário;
\end{itemize}

\subsection*{Objetivos Científicos}

\begin{itemize}
\item \textbf{OC1} - Criação de um catalogação sobre os principais mecanismos formais de participação, a fim de fazer um levantamento inicial para criação da biblioteca digital.
\item \textbf{OC2} - Um estudo inicial sobre a participação social, funcionamento dos mecanismos formais de participação presenciais e virtuais, além de exemplos sobre mecanismos de participação virtual já consolidados.
\item \textbf{OC3} - Um estudo sobre o funcionamento dos meios de participação social presentes no Participa.br.
\item \textbf{OC5} - Análise das ferramentas livres para desenvolvimento de bibliotecas digitais, a fim de comprovar qual a melhor ferramenta que atende as necessidades levantadas por esse trabalho.
\end{itemize}
	 	

\section{Metodologia}

Do ponto de vista metodológico, foi feito um estudo inicial sobre o tema na literatura concernente (em especial nas publicações disponibilizadas pelo IPEA) e nos conteúdos disponibilizados nos sites dos canais de participação social. Além disso, foram feitas entrevistas e reuniões com alguns gestores envolvidos para que pudesse ser obtida uma visão mais clara sobre o assunto e sobre as necessidades de informação.

A fim de manter uma solução mais duradoura para o problema da catalogação e organização de conteúdos, foi percebido que a solução mais adequada é a proposição de uma biblioteca digital, na qual as informações que eventualmente sejam alteradas num determinado Conselho, Conferência ou Ouvidoria possam ser automaticamente compartilhadas por todos e devidamente atualizadas no portal Participa.br.

Em seguida foram feitos estudos sobre plataformas de software livre voltadas para bibliotecas digitais e foram encontradas algumas soluções interessantes que podem ser adotadas no projeto, as quais estão listadas no Anexo \ref{Att:anexobibliotecas} deste documento. Além disso, como prova de conceito, foi adotado um desses softwares para demonstração de uma possível biblioteca digital construída com um desses produtos e com uma amostra de documentos de participação social (conteúdos listados no Anexo \ref{Att:catalogomecanismos}).

A fim de integrar a solução de organização de conteúdos com as consultorias do termo de referência BRA/12/018, foi feita uma proposta para ampliação da biblioteca digital num espaço semântico e social colaborativo, cujas ideias iniciais estão descritas no capítulo \ref{cap:extbibparticipa}.

\section{Organização do Trabalho}

Este trabalho está dividido dentro de 7 outros capítulos. 

No \ref{cap:sociedadepsocial}º capítulo é apresentado um breve histórico da sociedade civil e como ela influencia nas decisões dos governante, um breve histórico da participação social no Brasil, assim como os mecanismos formais de participação instituídos pelo Decreto 8243/2014;

No \ref{cap:participabr}º capítulo será apresentado o Participa.br, a plataforma virtual para participação social. Além de uma motivação para utilização de um software livre pelo Participa.br, assim como as ferramentas de participação social.

No \ref{cap:noosfero}º capitulo aborda um estudo sobre a arquitetura e funcionamento do Noosfero, a plataforma livre para criação de redes sociais que foi utilizado pelo Participa.

Já no \ref{cap:bibinterop}º capítulo contém um estudo sobre bibliotecas digitais e interoperabilidade de dados, que serão utilizados para implementação da biblioteca.

No \ref{cap:bibparticipabr}º capítulo é apresentado o primeiro protótipo da biblioteca digital de participação social com a utilização de todas as informações levantadas.

No \ref{cap:extbibparticipa}º capítulo é proposto uma ampliação das funcionalidades do portal Participa.br para torná-lo um espaço colaborativo de participação social semântico, além de monitoramento de políticas públicas.

O trabalho se encerra com os resultados obtidos nessa primeira fase, assim como as restrições encontradas. Também é feita uma análise sobre os levantamentos, além de uma discussão e cronograma para o prosseguimento do trabalho.