\chapter{Introdução}


\section{Objetivos}

\subsection{Objetivos Gerais}

O objetivo deste trabalho de conclusão de curso é o desenvolvimento de um ambiente de consulta colaborativa, que funcione de maneira descentralizada, onde cada instância é responsável pela manutenção de suas informações e que essas informações sejam disponibilizadas de maneira centralizada. 

\subsection{Objetivos Específicos}
	 	

\section{Metodologia}


\section{Organização do Trabalho}

Este trabalho está dividido dentro de 7 outros capítulos. 

Já no \ref{cap:bibinterop}º capítulo contém um estudo sobre bibliotecas digitais e interoperabilidade de dados, que serão utilizados para implementação da biblioteca.

No \ref{cap:noosfero}º capitulo aborda um estudo sobre a arquitetura e funcionamento do Noosfero, a plataforma livre para criação de redes sociais que foi utilizado pelo Participa.

No \ref{cap:bibparticipabr}º capítulo é apresentado a biblioteca digital de participação social com a utilização de todas as informações levantadas.

O trabalho se encerra com os resultados obtidos nessa primeira fase, assim como as restrições encontradas. Também é feita uma análise sobre os levantamentos, além de uma discussão e cronograma para o prosseguimento do trabalho.
