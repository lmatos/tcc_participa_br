\begin{resumo}

Este trabalho propõe-se para a construção de uma biblioteca digital distribuída, tomando o portal Participa.br, que utiliza o software livre Noosfero, como um provedor de serviços e os mecanismos formais de participação social como provedores de dados, cuja interoperabilidade ocorre por meio do protocolo OAI-PMH.
%Resumo tem apenas um parágrafo
%
Além da identificação da proposição citada, foram gerados os seguintes resultados preliminares: (i) uma abordagem teórica sobre participação social; (ii) uma introdução sobre o Participa.br; (iii) um explicativo sobre a ferramenta Noosfero, núcleo do portal Participa.br; (iv) uma prova de conceito na forma de uma biblioteca digital (centralizada) experimental, alimentada por uma amostra de documentos (v) uma compilação sobre produtos de software livre e soluções tecnológicas para a construção de bibliotecas digitais; (vi) uma proposta de ampliação das funcionalidades do portal Participa.br para torná-lo um espaço colaborativo de participação social com opção de monitoramento de informações sobre políticas públicas; (vii) um catalogo contendo os mecanismos formais levantados utilizados para um primeiro protótipo.


\textbf{Palavras-chave}: bibliotecas digitais, participação social, redes sociais, bibliotecas digitais semânticas.

\end{resumo}
