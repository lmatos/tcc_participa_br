\chapter{Catalogo de Mecanismos Formais de Pariticipação}
\label{Att:catalogomecanismos}

Neste anexo, contém o catalogo de alguns mecanismos formais de participação, utilizados para levantamento dos dados.

\section*{Conselhos}

%------------------------------------------------------------------------------
\subsection*{Conselho Curador do FGTS (CCFGTS)}

Link de acesso: http://www.fgts.gov.br/quem\_administra.asp

\subsubsection*{Objetivo}

O Conselho Curador do Fundo de Garantia do Tempo de Serviço (CCFGTS) é um conselho que gere e administra o Fundo de Garantia do Tempo de Serviço (FGTS). Atualmente é formado por um colegiado formado por entidades representativas dos trabalhadores, dos empregadores e representantes do Governo Federal.

\subsubsection*{Atribuições}

De acordo com a Lei nº 8.036, de 11 de maio de 1990,
são dadas as seguintes atribuições para o Conselho Curador do
FGTS:

\begin{enumerate}
\item 
Estabelecer as diretrizes e os programas de alocação de todos os
recursos do FGTS, de acordo com os critérios definidos nesta lei, em
consonância com a política nacional de desenvolvimento urbano e as
políticas setoriais de habitação popular, saneamento básico e
infraestrutura urbana estabelecida pelo Governo Federal;
\item 
Acompanhar e avaliar a gestão econômica e financeira dos recursos,
bem como os ganhos sociais e o desempenho dos programas aprovados;
\item 
Apreciar e aprovar os programas anuais e plurianuais do FGTS;
\item 
Pronunciar-se sobre as contas do FGTS, antes do seu encaminhamento aos
órgãos de controle interno para os fins legais;
\item 
Adotar as providências cabíveis para a correção de atos e
fatos do Ministério da Ação Social e da Caixa Econômica
Federal, que prejudiquem o desempenho e o cumprimento das finalidades
no que concerne aos recursos do FGTS;
\item 
Dirimir dúvidas quanto à aplicação das normas regulamentares,
relativas ao FGTS, nas matérias de sua competência;
\item 
Aprovar seu regimento interno;
\item 
Fixar as normas e valores de remuneração do agente operador e dos
agentes financeiros;
\item 
Fixar critérios para parcelamento de recolhimentos em atraso;
\item 
Fixar critério e valor de remuneração para o exercício da
fiscalização;
\item 
Divulgar, no Diário Oficial da União, todas as decisões proferidas
pelo Conselho, bem como as contas do FGTS e os respectivos pareceres
emitidos.
\end{enumerate}

\subsubsection*{Legislação}

Atualmente as leis que regem o CCFGTS, são as seguintes:

\begin{itemize}
\item Lei nº 8.036, de 11 de maio de 1990.
\item Decreto nº 6.827, de 22 de abril de 2009 (aumento no
número de representantes, ampliando o número de representantes do
governo e sociedade civil).
\end{itemize}

\subsubsection*{Composição}

De acordo com a lei 8.036, o órgão é um conselho tripartite
composto de dezesseis (16) membros, entre esses: representantes,
membros do governo empregadores e dos trabalhadores. A partir de um
decreto que é citado acima, o órgão passou a ter vinte e quatro
membros (24), dentre eles:

\begin{itemize}
\item 
Doze (12) representantes do governo.
\item 
Seis (6) representantes dos trabalhadores.
\item 
Seis (6) representantes dos empregadores.
\end{itemize}

%------------------------------------------------------------------------------
\newpage
\subsection*{O Conselho Curador da Empresa Brasil de Comunicação (EBC)}

Link de acesso: http://www.conselhocurador.ebc.com.br/

\subsubsection*{Objetivo}

O Conselho Curador da Empresa Brasil de Comunicação é
responsável gerir o cumprimento das atribuições da Empresa
Brasileira de Comunicação (EBC) e por determinar a linha
editorial de seus veículos. Apresentando e analisando sugestões
para a linha editorial e programação.

\subsubsection*{Atribuições}

Segundo o Art. 17 da Lei 11.652/08, compete ao conselho Curador da
empresa Brasil de Comunicação:

\begin{enumerate}
\item 
Deliberar sobre as diretrizes educativas, artísticas, culturais e
informativas integrantes da política de comunicação propostas
pela Diretoria Executiva da EBC;
\item 
Zelar pelo cumprimento dos princípios e objetivos previstos nesta
Lei;
\item 
Opinar sobre matérias relacionadas ao cumprimento dos princípios e
objetivos previstos nesta Lei;
\item 
Deliberar sobre a linha editorial de produção e programação
proposta pela Diretoria Executiva da EBC e manifestar-se sobre sua
aplicação prática;
\item 
Encaminhar ao Conselho de Comunicação Social as deliberações
tomadas em cada reunião;
\item 
Deliberar, pela maioria absoluta de seus membros, quanto à
imputação de voto de desconfiança aos membros da Diretoria
Executiva, no que diz respeito ao cumprimento dos princípios e
objetivos desta Lei;
\item 
Eleger o seu Presidente, dentre seus membros.
\end{enumerate}

\subsubsection*{Legislação}

A Lei que rege o Conselho Curador da Empresa Brasil de Comunicação
é composta pelo Artigo 17 da Lei 11.652/08.

\subsubsection*{Composição}

O Conselho é formado 22 membros, sendo composto por:

\begin{itemize}
\item 
Dois (2) representantes do governo (Congresso nacional) - \ Câmara dos
Deputados e Senado.
\item 
Quinze (15) representantes da sociedade
\item 
Quatro ministros -  Educação; Cultura; Secretaria de
Comunicação Social da Presidência da República; e Ciência
Tecnologia e Inovação.
\end{itemize}

%------------------------------------------------------------------------------
\newpage
\subsection*{Conselho Nacional das Cidades (Concidades)}

Link de acesso: http://www.cidades.gov.br/

\subsubsection*{Objetivo}

Este conselho faz parte do Ministério da Cidade e tem como objetivo
criar um canal diálogo de melhorias e sugestões de políticas
relacionadas às questões urbanísticas.

\subsubsection*{Legislação}

Foi instituído em 2003, através da Medida Provisória
nº 103. Atualmente é regido pela Lei 10.683/03.

\subsubsection*{Composição}

Esse conselho é composto por 86 ~membros de diversas classes
políticas e sociais, entre eles:

\begin{itemize}
\item 
Dezesseis (16) membros do poder público federal;
\item 
Nove (9) membros do poder público estadual;
\item 
Doze (12) membros do poder público municipal;
\item 
Vinte e três (23) membros representantes de entidades dos movimentos
populares;
\item 
Oito (8) membros de entidades dos empresários;
\item 
Oito (8) membros dos trabalhadores;
\item 
Seis (6) membros das entidades profissionais acadêmicas e de pesquisa
e
\item 
Quatro (4) de membros organizações não governamentais.
\end{itemize}

Podem ser inclusos mais 9 observadores representantes dos governos
estaduais, os quais tem que possuir um Conselho das Cidades em seu
estado.
%------------------------------------------------------------------------------
\newpage
\subsection*{Conselho Nacional de Ciência e Tecnologia (CCT)}

Link de acesso: http://www.mct.gov.br/index.php/content/view/10125.html

\subsubsection*{Atribuições}

São de atribuições do Conselho Nacional de Ciência e
Tecnologia:

\begin{enumerate}
\item 
Propor a política de Ciência e Tecnologia do País, como fonte e
parte integrante da política nacional de desenvolvimento;
\item 
Propor planos, metas e prioridades de governo referentes à Ciência e
Tecnologia, com as especificações de instrumentos e de recursos;
\item 
Efetuar avaliações relativas à execução da política
nacional de Ciência e Tecnologia;
\item 
Opinar sobre propostas ou programas que possam causar impactos à
política nacional de desenvolvimento científico e tecnológico,
bem como sobre atos normativos de qualquer natureza que objetivem
regulamentá-la.
\end{enumerate}

\subsubsection*{Legislação}

é regido pela Lei nº 9.257, de 09 de janeiro de
1996.

\subsubsection*{Composição}

é composto de 27 membros, conforme a lei que rege o CCT tem a seguinte
composição:

\begin{itemize}
\item 
Treze (13) representantes do governo federal, incluindo os ministros, da
Ciência e Tecnologia, Defesa e Casa Civil.
\item 
Oito (8) representantes da sociedade, sendo esses, produtores e
usuários da ciência e tecnologia.
\item 
Seis (6) representantes de entidades de ensino.
\end{itemize}
%------------------------------------------------------------------------------
\newpage
\subsection*{Conselho da Transparência Pública e Combate à Corrupção (CTPCC)}

Link de acesso: http://www.cgu.gov.br/ConselhoTransparencia/

\subsubsection*{Objetivo}

O CTPCC tem o objetivo de sugerir e debater melhorias em relação
à transparência na gestão da administração pública e
estratégias de combate à corrupção e à impunidade.

\subsubsection*{Atribuições}

Conselho da Transparência Pública e Combate à Corrupção
(CTPCC):

\begin{enumerate}
\item 
Contribuir para a formulação das diretrizes da política de
transparência da gestão de recursos públicos e de combate à
corrupção e à impunidade, a ser implementada pela
Controladoria-Geral da União e pelos demais órgãos e entidades da
administração pública federal;
\item 
Sugerir projetos e ações prioritárias da política de
transparência da gestão de recursos públicos e de combate à
corrupção e à impunidade;
\item 
Sugerir procedimentos que promovam o aperfeiçoamento e a
integração das ações de incremento da transparência e de
combate à corrupção e à impunidade, no âmbito da
administração pública federal;
\item 
Atuar como instância de articulação e mobilização da
sociedade civil organizada para o combate à corrupção e à
impunidade; e
\item 
Realizar estudos e estabelecer estratégias que fundamentem propostas
legislativas e administrativas tendentes a maximizar a transparência
da gestão pública e ao combate à corrupção e à
impunidade.
\end{enumerate}

\subsubsection*{Legislação}

é regido pelo Decreto nº 4.923 de 18 de Dezembro de
2003.

\subsubsection*{Composição}

Antigamente, o conselho era composto por 16 membros. A partir do Decreto
nº 7.857, de 6 de Dezembro de 2012, o conselho da
Transparência Pública e Combate à Corrupção passou a ser
compostos por 20 membros, divididos entre:

\begin{itemize}
\item Dez membros do Poder Executivo Federal.
\item Duas autoridades públicas que são convidadas
\item Oito membros da sociedade civil.
\end{itemize}
%------------------------------------------------------------------------------
\newpage
\subsection*{Conselho de Defesa dos Direitos da Pessoa Humana (CDDPH)}

Link de acesso: http://www.sdh.gov.br/sobre/participacao-social/cddph 

\subsubsection*{Objetivo}

Este conselho é vinculado à Secretaria de Direitos Humanos da
Presidência da República e tem como objetivo responder as
violações dos direitos humanos.

\subsubsection*{Atribuições}

Suas principais atribuições são:

\begin{enumerate}
\item 
Promover inquéritos, investigações e estudos acêrca da
eficácia das normas asseguradoras dos direitos da pessoa humana,
inscritos na Constituição Federal, na Declaração Americana
dos Direitos e Deveres Fundamentais do Homem (1948) e na
Declaração Universal dos Direitos Humanos (1948);
\item 
Promover a divulgação do conteúdo e da significação de
cada um dos direitos da pessoa humana mediante conferências e debates
em universidades, escolas, clubes, associações de classe e
sindicatos e por meio da imprensa, do rádio, da televisão, do
teatro, de livros e folhetos;
\item 
Promover inquéritos e investigações nas áreas onde tenham
ocorrido fraudes eleitorais de maiores proporções, para o fim de
sugerir as medidas capazes de escoimar de vícios os pleitos futuros;
\item 
Promover a realização de cursos diretos ou por correspondência
que concorram, para o aperfeiçoamento dos serviços policiais,
no que concerne ao respeito dos direitos da pessoa humana;
\item 
Promover entendimentos com os governos dos Estados e Territórios cujas
autoridades administrativas ou policiais se revelem, no todo ou em
parte, incapazes de assegurar a proteção dos direitos da pessoa
humana para o fim de cooperar com os mesmos na reforma dos respectivos
serviços e na melhor preparação profissional e cívica dos
elementos que os compõem;
\item 
Promover entendimentos com os governos estaduais e municipais e com a
direção de entidades autárquicas e de serviços autônomos,
que estejam por motivos poIíticos, coagindo ou perseguindo seus
servidores, por qualquer meio, inclusive transferências,
remoções e demissões, a fim de que tais abusos de poder não
se consumem ou sejam, afinal, anulados;
\end{enumerate}

\subsubsection*{Legislação}

Foi criado pela Lei nº 4.319, de 16 de março de
1964. Foram feitas alterações através da Lei
nº 5.763, de 15 de dezembro de 1971, e pela Lei
nº 10.683, de 28 de maio de 2003.
%------------------------------------------------------------------------------
\newpage
\subsection*{Conselho de Desenvolvimento Econômico e Social (CDES)}

Link de acesso: http://www.cdes.gov.br/

\subsubsection*{Objetivo}

é um órgão de caráter consultivo da Presidência da
república, que tem como objetivo a formação do juízo
político do Governo, através do estabelecimento de um diálogo
para a discussão de políticas públicas e propor medidas
necessárias para alavancar o crescimento do país

\subsubsection*{Atribuições}

De acordo com a Lei nº 10.683, de 28 de Maio de 2003,
compete ao Conselho de Desenvolvimento Econômico e Social:
assessorar o Presidente da República na
formulação de políticas e diretrizes específicas, voltadas ao
desenvolvimento econômico e social, produzindo indicações
normativas, propostas políticas e acordos de procedimento, e apreciar
propostas de políticas públicas e de reformas estruturais e de
desenvolvimento econômico e social que lhe sejam submetidas pelo
Presidente da República, com vistas na articulação das
relações de governo com representantes da sociedade civil
organizada e no concerto entre os diversos setores da sociedade nele
representados.

\subsubsection*{Legislação}

é regido pela Lei nº 10.683, de 28 de maio de 2003.

\subsubsection*{Composição}

O conselho de Desenvolvimento Econômico e Social é composto por:

\begin{itemize}
\item 
Um Presidente do Conselho de Desenvolvimento Econômico e Social
\item 
Um Secretário Executivo
\item 
Um comitê gestor, composto por seis (6) membros, que tem como objetivo
gerir o Conselho de Desenvolvimento Econômico e Social
\item 
Noventa e seis (96) membros da sociedade civil, composto por professores
de instituições de ensino, presidente de instituições,
união nacional dos estudantes, entre outros.
\item 
Dezoito (18) membros do governo (Ministros)
\end{itemize}
%------------------------------------------------------------------------------
\newpage
\subsection*{Conselho Nacional de Desenvolvimento Rural Sustentável (Condraf)}

Link de acesso: http://portal.mda.gov.br/portal/condraf/

\subsubsection*{Objetivo}

Criado em 1999, tem o objetivo de contribuir para redução da
desigualdade social e combate a pobreza, por meio de geração de
emprego e renda, além de propor a adoção de instrumentos de
participação e controle social nas fases de planejamento e
execução de políticas públicas para o desenvolvimento rural
sustentável.

\subsubsection*{Atribuições}

As principais atribuições do Condraf são:

\begin{enumerate}
\item 
Contribuir para a superação da pobreza por meio da geração
de emprego e renda;
\item 
Contribuir para a redução das desigualdades de renda, gênero,
geração e etnia;
\item 
Colaborar para a diversificação das atividades econômicas e sua
articulação dentro e fora de territórios rurais;
\item 
Propiciar a adoção de instrumentos de participação e
controle social nas fases de planejamento e execução de
políticas públicas para o desenvolvimento rural sustentável.
\end{enumerate}

\subsubsection*{Legislação}

O Condraf foi criado e é regido pelo Decreto nº-
4.854, de 8 de outubro de 2003, o qual diz sua composição,
estruturação, competências e funcionamento.
%------------------------------------------------------------------------------
\newpage
\subsection*{Conselho Nacional de Defesa Civil (Conpdec)}

Link de acesso:
http://www.integracao.gov.br/web/guest/apresentacao-conpdec

\subsubsection*{Objetivo}

é um órgão consultivo que tem como objetivo planejar ações
de prevenção a desastres de maiores prevalência no país,
realizar estudos para reduzir riscos de desastres, minizar danos e
atuar quando houver um desastre.


\subsubsection*{Atribuições}

As atribuições do Conselho Nacional de Defesa Civil, segundo
consta no site do mesmo são as seguintes:

\begin{itemize}
\item 
Planejar e promover ações de prevenção de desastres
naturais, antropogênicos e mistos, de maior prevalência no País;
\item 
Realizar estudos, avaliar e reduzir riscos de desastres;
\item 
Atuar na iminência e em circunstâncias de desastres; e
\item 
Prevenir ou minimizar danos, socorrer e assistir populações
afetadas, e restabelecer os cenários atingidos por desastres.
\end{itemize}

\subsubsection*{Legislação}

é regido pela Lei 12.608 de 10 de abril de 2012.
%------------------------------------------------------------------------------
\newpage
\subsection*{Conselho Nacional de Economia Solidária (CNES)}

Link de acesso: http://www3.mte.gov.br/ecosolidaria/cons\_default.asp

\subsubsection*{Objetivo}

Criado em junho de 2003, tem como objetivo a propor diretrizes para as
ações relacionada à economia solidária nos Ministérios que
o integram e em outros órgãos do Governo Federal, além do
acompanhamento da execução destas ações, no âmbito de uma
política nacional de economia solidária.

\subsubsection*{Atribuições}

\begin{enumerate}
\item 
Estimular a participação da sociedade civil e do Governo no
âmbito da política de economia solidária;
\item 
Propor diretrizes e prioridades para a política de economia
solidária;
\item 
Propor medidas para o aperfeiçoamento da legislação, com
vistas ao fortalecimento da economia solidária;
\item 
Avaliar o cumprimento dos programas da Secretaria Nacional de Economia
Solidária do Ministério do Trabalho e Emprego e sugerir medidas
para aperfeiçoar o seu desempenho;
\item 
Examinar propostas de políticas públicas que lhe forem submetidas
pela Secretaria Nacional de Economia Solidária;
\item 
Coordenar as atividades relacionadas com a economia solidárias
desenvolvidas pelas entidades nele representadas com as da Secretaria
Nacional de Economia Solidária;
\item 
Estimular a formação de novas parcerias entre as entidades nele
representadas e a Secretaria Nacional de Economia Solidária;
\item 
Colaborar com os demais conselhos envolvidos com as políticas
públicas de desenvolvimento, combate ao desemprego e à pobreza; e
\item 
Aprovar o seu regimento interno.
\end{enumerate}

\subsubsection*{Legislação}

é regido pelo Decreto nº 5811, de 21 de junho de
2006 que diz sobre sua estruturação, composição e
funcionamento.

\subsubsection*{Composição}

Atualmente o Conselho Nacional de Economia Solidária (CNES) é um
conselho tripartite, que possui membros das seguinte classes:

\begin{itemize}
\item 
Governo Federal, Secretarias Estaduais de Trabalho
\item 
Empreendimentos Econômicos Solidários:
\item 
Outras organizações da Sociedade Civil e Serviços Sociais:
\end{itemize}
%------------------------------------------------------------------------------
\newpage
\subsection*{Conselho Nacional de Educação (CNE)}

Link de acesso: http://portal.mec.gov.br/cne/


\subsubsection*{Objetivo}


Criado em 1995, tem por missão a busca democrática de alternativas e
mecanismos institucionais que possibilitem, no âmbito de sua esfera
de competência, assegurar a participação da sociedade no
desenvolvimento, aprimoramento e consolidação da educação
nacional de qualidade. (Retirado do site).


\subsubsection*{Atribuições}


Segundo a Lei n. 9.131 de abril de 2012, compete ao conselho nacional de
educação:

\begin{enumerate}
\item 
Subsidiar a elaboração e acompanhar a execução do Plano
Nacional de Educação;
\item 
Manifestar-se sobre questões que abranjam mais de um nível ou
modalidade de ensino;
\item 
Assessorar o Ministério da Educação e do Desporto no
diagnóstico dos problemas e deliberar sobre medidas para
aperfeiçoar os sistemas de ensino, especialmente no que diz
respeito à integração dos seus diferentes níveis e
modalidades;
\item 
Emitir parecer sobre assuntos da área educacional, por iniciativa de
seus conselheiros ou quando solicitado pelo Ministro de Estado da
Educação e do Desporto;
\item 
Manter intercâmbio com os sistemas de ensino dos Estados e do Distrito
Federal;
\item 
Analisar e emitir parecer sobre questões relativas à aplicação
da legislação educacional, no que diz respeito à
integração entre os diferentes níveis e modalidade de ensino;
\item 
Elaborar o seu regimento, a ser aprovado pelo Ministro de Estado da
Educação e do Desporto.
\end{enumerate}

\subsubsection*{Legislação}

Foi criado pela Lei n. 9.131, de 24 de novembro de 1995.

\subsubsection*{Composição}


Um membro do Conselho Nacional de Educação preside o conselho, e
é composto pelos seguintes membros:

%------------------------------------------------------------------------------
\newpage
\subsection*{Conselho Nacional de Imigração (CNIg)}

Link de acesso: http://portal.mte.gov.br/cni/


\subsubsection*{Objetivo}

Criado em 1980, tem como objetivo formular a política de
imigração, auxiliar imigrantes através do levantamento da
mão-de-obra estrangeira qualificada, promover estudos relacionadas as
políticas de imigração, entre outros.


\subsubsection*{Atribuições}


Segundo o Decreto nº 840, de 22 de junho de 1993, o Conselho
Nacional de Imigração é responsável por:

\begin{enumerate}
\item 
Formular a política de imigração;
\item 
Coordenar e orientar as atividades de imigração;
\item 
Efetuar o levantamento periódico das necessidades de mão-de-obra
estrangeira qualificada, para admissão em caráter permanente ou
temporário;
\item 
Definir as regiões de que trata o art. 18 da Lei nº 6.815,
de 19 de agosto de 1980, e elaborar os respectivos planos de
imigração;
\item 
Promover ou fornecer estudos de problemas relativos à
imigração;
\item 
Estabelecer normas de seleção de imigrantes, visando proporcionar
mão-de-obra especializada aos vários setores da economia nacional e
captar recursos para setores específicos;
\item 
Dirimir as dúvidas e solucionar os casos omissos, no que diz respeito
a imigrantes;
\item 
Opinar sobre alteração da legislação relativa à
imigração, quando proposta por qualquer órgão do Poder
Executivo;
\item 
Elaborar seu regimento interno, que deverá ser submetido à
aprovação do Ministro de Estado do Trabalho.
\end{enumerate}

\subsubsection*{Composição}


O conselho é formado por representantes:

\begin{itemize}
\item 
Dos seguintes Ministérios: do Trabalho e Emprego (responsável pela
presidência do conselho), das Relações Exteriores, da
Agricultura e Abastecimento, da Ciência e Tecnologia, do
Desenvolvimento, Indústria e Comércio Exterior, da Saúde e da
Educação.
\item 
Cinco membros da sociedade civil (trabalhadores)
\item 
Cinco membros dos empregadores
\item 
Um representante da comunidade cientifica e tecnológica.
\end{itemize}
%------------------------------------------------------------------------------
\newpage
\subsection*{Conselho Nacional de Política Criminal e Penitenciária (CNPCP)}

Link de acesso: http://portal.mj.gov.br/cnpcp/

\subsubsection*{Objetivo}

Criado em junho de 1980, tem como objetivo gerar informações, de
análises, de deliberações e de estímulo intelectual e
material às atividades de prevenção da criminalidade.

\subsubsection*{Atribuições}


As atribuições do Conselho Nacional de Política Criminal e
Penitenciária, foram atribuidas através do Decreto
nº 6.061, de 15 de março de 2007 e são as
seguintes:

\begin{enumerate}
\item 
Propor diretrizes da política criminal quanto à prevenção do
delito, administração da Justiça Criminal e execução
das penas e das medidas de segurança;
\item 
Contribuir na elaboração de planos nacionais de desenvolvimento,
sugerindo as metas e prioridades da política criminal e
penitenciária;
\item 
Promover a avaliação periódica do sistema criminal para a sua
adequação às necessidades do País;
\item 
Estimular e promover a pesquisa criminologia;
\item 
Elaborar programa nacional penitenciário de formação e
aperfeiçoamento do servidor;
\item 
Estabelecer regras sobre a arquitetura e construção de
estabelecimentos penais e casas de albergados;
\item 
Estabelecer os critérios para a elaboração da estatística
criminal;
\item 
Inspecionar e fiscalizar os estabelecimentos penais, bem assim
informar-se, mediante relatórios do Conselho Penitenciário,
requisições, visitas ou outros meios, acerca do desenvolvimento
da execução penal nos Estados e Distrito Federal, propondo às
autoridades dela incumbida as medidas necessárias ao seu
aprimoramento;
\item 
Representar ao Juiz da Execução ou à autoridade administrativa
para instauração de sindicância ou procedimento administrativo,
em caso de violação das normas referentes à execução
penal; e
\item 
Representar à autoridade competente para a interdição, no todo
ou em parte, de estabelecimento penal
\end{enumerate}

\subsubsection*{Legislação}


Possui competências atribuídas pelo Decreto nº
6.061, de 15 de março de 2007.


\subsubsection*{Composição}


O conselho é composto de 13 membros, designados pelo Ministério da
Justiça. Dentre esses: profissionais do direito penal e
judiciário, membros da sociedade cívil, professores e membros do
governo (Ministérios).

%------------------------------------------------------------------------------
\newpage
\subsection*{Conselho Nacional de Política Cultural (CNPC)}

Link de acesso: http://www.cultura.gov.br/cnpc/


\subsubsection*{Objetivo}


é um órgão vinculado ao Ministério da Cultura do Brasil, que tem
como objetivo atuar na proposição, avaliação e
fiscalização de políticas públicas de cultura.

\subsubsection*{Atribuições}

Através do Decreto nº 5.520, de 24 de Agosto de
2005, de 19 de novembro de 1992, o Conselho Nacional de Política
Cutural, recebeu as suas atribuições:


\subsubsection*{Legislação}


é regido pelo Decreto nº 5.520, de 24 de Agosto de
2005, que institui o seu funcionamento, atribuições e
composição.


\subsubsection*{Composição}

\url{http://www.planalto.gov.br/ccivil_03/_ato2004-2006/2005/Decreto/D5520.htm}
%------------------------------------------------------------------------------
\newpage
\subsection*{Conselho Nacional de Políticas sobre Drogas (Conad)}

Link de acesso: http://www.obid.senad.gov.br/portais/CONAD/


\subsubsection*{Objetivo}



órgão máximo, vinculado ao Ministério da Justiça e superior
ao Sistema Nacional de Políticas Públicas sobre Drogas, que tem
como objetivo regulamentar, deliberar e pesquisar o uso de
substâncias químicas e determina a classificação de quais
substancias são drogas e quais não são. é responsável pela
realização de campanhas para esclarecimento quanto às drogas.


\subsubsection*{Atribuições}


Compete ao Conselho Nacional de Políticas sobre Drogas, as seguinte
atribuições:

\begin{enumerate}
\item 
Acompanhar e atualizar a política nacional sobre drogas, consolidada
pela SENAD;
\item 
Exercer orientação normativa sobre as atividades previstas no art.
1o;
\item 
Acompanhar e avaliar a gestão dos recursos do Fundo Nacional
Antidrogas - FUNAD e o desempenho dos planos e programas da política
nacional sobre drogas;
\item 
Propor alterações em seu Regimento Interno; e
\item 
Promover a integração ao SISNAD dos órgãos e entidades
congêneres dos Estados, dos Municípios e do Distrito Federal.
\end{enumerate}


\subsubsection*{Legislação}

Decreto nº 5.912, de 27 de Setembro de 2006.

\subsubsection*{Composição}


http://www.planalto.gov.br/ccivil\_03/\_ato2004-006/2006/Decreto/D5912.htm

%------------------------------------------------------------------------------
\newpage
\subsection*{Conselho Nacional de Previdência Social (CNPS)}

Link de acesso:
http://www.previdencia.gov.br/a-previdencia/orgaos-colegiados/
conselho-nacional-de-previdencia-social-cnps/.


\subsubsection*{Objetivo}



é responsável por tratar o direito previdenciário, ou seja, o
conselho tem como objetivo melhorar desempenho dos serviços
prestados à clientela previdenciária.


\subsubsection*{Atribuições}


De acordo com a Lei Nº 8.213, de 24 de Julho de 1991
compete ao \ conselho nacional de previdência social:

\begin{enumerate}
\item 
Estabelecer diretrizes gerais e apreciar as decisões de políticas
aplicáveis à Previdência Social;
\item 
Participar, acompanhar e avaliar sistematicamente a gestão
previdenciária;
\item 
Apreciar e aprovar os planos e programas da Previdência Social;
\item 
Apreciar e aprovar as propostas orçamentárias da Previdência
Social, antes de sua consolidação na proposta orçamentária
da Seguridade Social;
\item 
\ Acompanhar e apreciar, através de relatórios gerenciais por ele
definidos, a execução dos planos, programas e orçamentos no
âmbito da Previdência Social;
\item 
\ Acompanhar a aplicação da legislação pertinente à
Previdência Social;
\item 
Apreciar a prestação de contas anual a ser remetida ao Tribunal de
Contas da União, podendo, se for necessário, contratar auditoria
externa;
\item 
Estabelecer os valores mínimos em litígio, acima dos quais será
exigida a anuência prévia do Procurador-Geral ou do Presidente do
INSS para formalização de desistência ou transigência
judiciais, conforme o disposto no art. 132;
\item 
Elaborar e aprovar seu regimento interno.
\end{enumerate}

\subsubsection*{Legislação}


A Lei nº 8.213, de 24 de Julho de 1991 delibera sobre
o funcionamento, composição e atribuições do Conselho
Nacional de Previdência Social.


\subsubsection*{Composição}


Segundo a Lei nº 8.213, de 24 de Julho de 1991, o
orgão é composto por vários membros do governo/sociedade civil,
dentre eles, podemos destacar:

\begin{itemize}
\item 
Seis representantes do governo Federal
\item 
Nove membros da sociedade civil, dentre eles:
\item 
Três representantes dos aposentados ou pensionistas.
\item 
Três representantes dos trabalhadores.
\item 
Três representantes dos empregadores.
\end{itemize}
%------------------------------------------------------------------------------
\newpage
\subsection*{Conselho Nacional de Promoção da Igualdade Racial (CNPIR)}

Link de acesso: http://www.seppir.gov.br/apoiproj


\subsubsection*{Objetivo}

Esse órgão tem como objetivo propor, em âmbito nacional,
políticas de promoção da igualdade racial com ênfase na
população negra e outros segmentos raciais e étnicos da
população brasileira.


\subsubsection*{Atribuições}

Participar na elaboração de
critérios e parâmetros para a formulação e
implementação de metas e prioridades para assegurar as
condições de igualdade à população negra e de outros
segmentos étnicos da população brasileira;

\begin{enumerate}
\item Propor estratégias de acompanhamento, avaliação e
fiscalização, bem como a participação no processo
deliberativo de diretrizes das políticas de promoção da
igualdade racial, fomentando a inclusão da dimensão racial nas
políticas públicas desenvolvidas em âmbito nacional;
\item Apreciar anualmente a proposta orçamentária da Secretaria Especial
de Políticas de Promoção da Igualdade Racial e sugerir
prioridades na alocação de recursos;
\item Apoiar a Secretaria Especial de Políticas de Promoção da
Igualdade Racial na articulação com outros órgãos da
administração pública federal e os governos estadual, municipal
e do Distrito Federal;
\item Recomendar a realização de estudos, debates e pesquisas sobre a
realidade da situação da população negra e de outros
segmentos étnicos da população brasileira, com vistas a
contribuir na elaboração de propostas de políticas públicas
que visem à promoção da igualdade racial e à eliminação
de todas as formas de preconceito e discriminação;
\item Propor a realização de conferências nacionais de promoção
da igualdade racial, bem como participar de eventos que tratem de
políticas públicas de interesse da população negra e de
outros segmentos étnicos da população brasileira;
\item Apresentar sugestões para a
elaboração do planejamento plurianual do Governo Federal, o
estabelecimento de diretrizes orçamentárias e a alocação de
recursos no Orçamento Anual da União, visando subsidiar
decisões governamentais relativas à implementação de
ações de promoção da igualdade racial;
\item Propor a realização e acompanhar
o processo organizativo da conferência nacional de promoção da
igualdade racial, bem como participar de eventos que tratem de
políticas públicas de interesse da população negra e de
outros segmentos étnicos da população brasileira;
\item Zelar pelas deliberações das conferências nacionais de
promoção da igualdade racial;
\end{enumerate}

\subsubsection*{Legislação}

Através do Decreto nº 4.885, de 20 de Novembro 2003,
que dispõe sobre a composição, estruturação,
competências e funcionamento do Conselho Nacional de Promoção
da Igualdade Racial - CNPIR.


\subsubsection*{Composição}


Podemos observar que no Decreto \ nº 4.885, ele
dispoõe sobre a composição do CNPIR, que é composto por:

\begin{itemize}
\item 
\ Vinte e dois (22) membros do Poder Público Federal.
\item 
Dezenove (19) membros da sociedade civil, escolhidas através de edital
público.
\item 
Três (3) notáveis indicados pela SEPPIR.
\end{itemize}
%------------------------------------------------------------------------------
\newpage
\subsection*{Conselho Nacional de Recursos Hídricos (CNRH)}

Link de acesso: http://www.cnrh.gov.br/


\subsubsection*{Objetivo}

é um órgão integrante do Sistema Nacional de Gerenciamento dos
Recursos Hídricos. Sua responsabilidade é gerenciar os recursos
hídricos brasileiros, através da articulação do planejamento
dos recursos hídricos com o planejamento nacional, estadual e setores
usuários.


\subsubsection*{Atribuições}

De acordo com a Lei nº 9433, de Janeiro de 1997, compete ao
Conselho Nacional de Recursos Hídricos:

\begin{enumerate}
\item 
Promover a articulação do planejamento de recursos hídricos com
os planejamentos nacional, regional, estaduais e dos setores
usuários;
\item 
Arbitrar, em última instância administrativa, os conflitos
existentes entre Conselhos Estaduais de Recursos Hídricos;
\item 
Deliberar sobre os projetos de aproveitamento de recursos hídricos
cujas repercussões extrapolem o âmbito dos Estados em que serão
implantados;
\item 
Deliberar sobre as questões que lhe tenham sido encaminhadas pelos
Conselhos Estaduais de Recursos Hídricos ou pelos Comitês de Bacia
Hidrográfica;
\item 
Analisar propostas de alteração da legislação pertinente a
recursos hídricos e à Política Nacional de Recursos Hídricos;
\item 
Estabelecer diretrizes complementares para implementação da
Política Nacional de Recursos Hídricos, aplicação de seus
instrumentos e atuação do Sistema Nacional de Gerenciamento de
Recursos Hídricos;
\item 
Aprovar propostas de instituição dos Comitês de Bacia
Hidrográfica e estabelecer critérios gerais para a elaboração
de seus regimentos;
\item 
Acompanhar a execução e aprovar o Plano Nacional de Recursos
Hídricos e determinar as providências necessárias ao cumprimento
de suas metas;
\item 
Estabelecer critérios gerais para a outorga de direitos de uso de
recursos hídricos e para a cobrança por seu uso.
\item 
Zelar pela implementação da Política Nacional de Segurança
de Barragens (PNSB);
\item 
Estabelecer diretrizes para implementação da PNSB, aplicação
de seus instrumentos e atuação do Sistema Nacional de
Informações sobre Segurança de Barragens (SNISB);
\item 
Apreciar o Relatório de Segurança de Barragens, fazendo, se
necessário, recomendações para melhoria da segurança das
obras, bem como encaminhá-lo ao Congresso Nacional.
\end{enumerate}

\subsubsection*{Legislação}

As seguintes Leis regem o Conselho Nacional dos Recursos Hídricos:

\begin{itemize}
\item 
Lei nº 9.433, de 8 de Janeiro de 1997 -- Dispõe
sobre a criação da Política Nacional de Recursos Hídricos,
cria o Sistema Nacional de Gerenciamento de Recursos Hídricos,
incluso a criação do Conselho Nacional dos Recursos Hídricos,
assim como a sua composição e competências.
\item 
Lei nº 9.984, de 17 DE Julho de 2000 -- Apresenta
algumas competências para o Conselho.
\end{itemize}

\subsubsection*{Composição}

De acordo com a Lei nº 9.433, de 8 de Janeiro de 1997,
a composição do conselho Nacional de Recursos Hídricos é
Formada por

\begin{itemize}
\item 
Membros dos Ministérios e Secretarias da Presidência da República
com atuação no gerenciamento ou no uso de recursos hídricos;
\item 
Representantes indicados pelos Conselhos Estaduais de Recursos
Hídricos;
\item 
Representantes dos usuários dos recursos hídricos;
\item 
Representantes das organizações civis de recursos hídricos.
\end{itemize}
%------------------------------------------------------------------------------
\newpage
\subsection*{Conselho Nacional de Saúde (CNS)}

Link de acesso: http://conselho.saude.gov.br/

\subsubsection*{Objetivo}

é responsável por gerenciar a saúde no Brasil, deliberando,
fiscalizando, acompanhando e monitorando as políticas públicas de
saúde.

\subsubsection*{Atribuições}

Segundo o Decreto nº 5.839, de 11 de julho de 2006,
são competências do Conselho Nacional de Saúde:

\begin{enumerate}
\item 
Atuar na formulação de estratégias e no controle da
execução da Política Nacional de Saúde, na esfera do Governo
Federal, inclusive nos aspectos econômicos e financeiros;
\item 
Estabelecer diretrizes a serem observadas na elaboração dos planos
de saúde, em função das características epidemiológicas e
da organização dos serviços;
\item 
Elaborar cronograma de transferência de recursos financeiros aos
Estados, ao Distrito Federal e aos Municípios, consignados ao Sistema
\'Unico de Saúde - SUS;
\item 
Aprovar os critérios e os valores para remuneração de
serviços e os parâmetros de cobertura de assistência;
\item 
Propor critérios para a definição de padrões e parâmetros
assistenciais;
\item 
Acompanhar e controlar a atuação do setor privado da área da
saúde, credenciado mediante contrato ou convênio;
\item 
Acompanhar o processo de desenvolvimento e incorporação
científica e tecnológica na área de saúde, visando à
observação de padrões éticos compatíveis com o
desenvolvimento sócio-cultural do País; e
\item 
Articular-se com o Ministério da Educação quanto à
criação de novos cursos de ensino superior na área de saúde,
no que concerne à caracterização das necessidades sociais
\end{enumerate}

\subsubsection*{Legislação}

Decreto nº 5.839, de 11 de julho de 2006 - Dispõe
sobre a organização, as atribuições e o processo eleitoral
do Conselho Nacional de Saúde - CNS e dá outras providências.

\subsubsection*{Composição}

Segundo o Decreto nº 5.839, de 11 de julho de 2006, o
Conselho Nacional de Saúde é composto pelos seguintes integrantes:

\begin{enumerate}
\item Cinquenta por cento de representantes de entidades e dos movimentos
sociais de usuários do SUS; 
\item Cinquenta por cento de representantes de entidades de profissionais de
saúde, incluída a comunidade científica da área de saúde, de
representantes do governo, de entidades de prestadores de serviços
de saúde, do Conselho Nacional de Secretários de Saúde - CONASS,
do Conselho Nacional de Secretários Municipais de Saúde - CONASEMS
e de entidades empresariais com atividade na área de saúde.
\end{enumerate}

%------------------------------------------------------------------------------
\newpage
\section*{Conferências}

\subsection*{Conferência Nacional sobre Migrações e Refúgio (COMIGRAR)}

Link de acesso: http://www.participa.br/comigrar

\subsubsection*{Número de Versões}

Primeira versão que vai ser realizada em 2014.

\subsubsection*{Objetivo}

Segundo os documentos presentes no site oficial, o objetivo dessa
primeira Conferência Nacional sobre Migrações e Refúgio é
reunir migrantes, refugiados, profissionais envolvidos na temática
migratória, estudiosos, servidores públicos, representações
diversas que vivenciam a realidade da migração, para uma
reflexão e aporte coletivos de insumos para a Política e do Plano
Nacionais de Migrações e Refúgio.


\subsubsection*{Data da Conferência}

Essa conferência vai realizar-se no período entre 30 de maio a 01 de
junho na cidade de Brasília, DF.
%------------------------------------------------------------------------------
\newpage
\subsection*{Conferência Nacional de Direitos Humanos (CNDH)}

Link de acesso:
http://portal.mj.gov.br/sedh/11cndh/site/historico/historico.html

\subsubsection*{Número de Versões}


Doze (12) versões, onde a primeira foi realizada em 1996, até o ano
de 2004 em todos os anos houve uma conferência, totalizando nove, em
2006 houve mais uma edição. Em 2008 houve mais uma edição e
a última está prevista para dezembro de 2015.

\subsubsection*{Tema Principal}

Democracia, Desenvolvimento e Direitos Humanos:
Superando as Desigualdades!

\subsubsection*{Objetivo}

Segundo o site oficial, o principal objetivo da 11{\textordfeminine}
Conferência Nacional dos Direitos Humanos (que foi a última que
ocorreu) foi a revisão e atualização do Programa Nacional de
Direitos Humanos (PNDH), num processo pautado pela interação
democrática entre o governo e a sociedade civil.

\subsubsection*{Data da Conferência}

Essa conferência realizou-se no período entre 15 a 18 de dezembro de
2008 na cidade de Brasília, DF.

%------------------------------------------------------------------------------
\newpage
\subsection*{Conferência Nacional do Meio Ambiente (CNMA)}

Link de acesso: http://www.mma.gov.br/informma/item/8710-iv-cnma-foca-res\%C3\%ADduos-s\%C3\%B3lidos

\subsubsection*{Número de Versões}

Quatro (4) versões, onde a primeira foi realizada em 2003, a segunda
em 2005, terceira em 2008 e quarta versão em 2013.

\subsubsection*{Tema Principal}

Vamos cuidar do Brasil!

\subsubsection*{Objetivo}

O objetivo da 4º Conferência foi à discussão
sobre a Política Nacional de Resíduos Sólidos que se tornou uma
preocupação após a aprovação da Lei 12.305/2010.

\subsubsection*{Data da Conferência}

Essa conferência realizou-se no período entre 24 a 27 de novembro de
2013 na cidade de Brasília, DF.

%------------------------------------------------------------------------------
\newpage
\subsection*{Conferência Nacional das Cidades (CNC)}

Link de acesso: http://www.cidades.gov.br/5conferencia/

\subsubsection*{Número de Versões}

Cinco (5) versões, onde a primeira foi realizada em 2003, a segunda em
2005, terceira em 2007, quarta em 2010 e quinta versão é realizada
agora em 2013.

\subsubsection*{Tema Principal}

Quem muda a cidade somos nós: Reforma Urbana
já!


\subsubsection*{Objetivo}


Segundo consta no regimento interno da 5 conferência das cidades, os
principais objetivos dessa versão :


\begin{enumerate}
\item Avançar na construção da Política e do Sistema Nacional de
Desenvolvimento Urbano;
\item Indicar prioridades de atuação ao Ministério das Cidades, e;
\item Eleger as entidades nacionais membros do Conselho das Cidades, para o
triênio 2014/2016, conforme Resolução Normativa do Conselho das
Cidades.
\end{enumerate}

\subsubsection*{Data da Conferência}


Essa conferência realizou-se no período entre 20 a 24 de novembro de
2014 na cidade de Brasília, DF.
%------------------------------------------------------------------------------
\newpage
\section*{Fóruns}

\subsection*{Fórum Nacional de Educação}

\subsubsection*{Objetivos}

FNE é um espaço inédito de interlocução entre a sociedade
civil e o Estado brasileiro, reivindicação histórica da
comunidade educacional e fruto de deliberação da Conferência
Nacional de Educação (CONAE - 2010).


\subsubsection*{Atribuições}


\begin{enumerate}
\item 
Participar do processo de concepção, implementação e
avaliação da política nacional de educação;
\item 
Acompanhar, junto ao Congresso Nacional, a tramitação de projetos
legislativos referentes à política nacional de educação, em
especial a de projetos de leis dos planos decenais de educação
definidos na Emenda à Constituição 59/2009;
\item 
Acompanhar e avaliar os impactos da implementação do Plano
Nacional de Educação;
\item 
Acompanhar e avaliar o processo de implementação das
deliberações das conferências nacionais de educação;
\item 
Elaborar seu Regimento Interno e aprovar ad
referendum o Regimento Interno das conferências
nacionais de educação;
\item 
Oferecer suporte técnico aos Estados, Municípios e Distrito Federal
para a organização de seus fóruns e de suas conferências de
educação;
\item 
Zelar para que os fóruns e as conferências de educação dos
Estados, do Distrito Federal e dos Municípios estejam articuladas à
Conferência Nacional de Educação;
\item 
Planejar e coordenar a realização de conferências nacionais de
educação, bem como divulgar as suas deliberações.


\subsubsection*{Legislação}
\item 
é regido por uma Portaria MEC \ n.º 1.407, de 14 de
dezembro de 2010, que foi pública no Diário Oficial da União em
dezembro de 2010.

\subsubsection*{Composição}
\item 
É composto por 35 membros, dentre eles, membros da sociedade civil e
do poder público.
\end{enumerate}
